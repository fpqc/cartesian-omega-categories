\section{Augmented directed complexes and the lax tensor product}
In this section, we will review Steiner's theory of augmented directed complexes \cite{steinerchain}, which is, in our opinion, the latest and greatest incarnation of the theory of pasting diagrams.  We will ultimately need to make use of a pasting theory in order to construct the lax tensor product of strict \(\omega\)-categories, which we will eventually transfer to our model category of weak \(\omega\)-categories.  The author makes no claims of originality in this section and directs all praise to Richard Steiner.
\subsection{Augmented directed complexes}
\begin{defn} An \dfn{augmented directed complex} is given by the following data:
\begin{enumerate}
\item[(i)] A connective (that is, \(0\) in negative degree) chain complex \((A_\bullet, \partial)\) of abelian groups
\item[(ii)] An augmentation map \(\varepsilon: A_0 \to \mathbf{Z}\) such that the composite \(A_1\xrightarrow{\partial}A_0 \xrightarrow{\varepsilon} \mathbf{Z}\) is \(0\).
\item[(iii)] A family of distinguished submonoids \(A^\ast_i\subseteq A_i\) for all \(i\geq 0\).
\end{enumerate}
A morphism of augmented directed complexes \[f_\bullet: (A_\bullet, \partial, \varepsilon, A^\ast_\bullet) \to (B_\bullet, \partial_B, \varepsilon_B, B^\ast_\bullet)\] is a chain map \(f_\bullet\) such that:
\begin{enumerate}
\item[(a)] \(\varepsilon_B \circ f_0 = \varepsilon_A\)
\item[(b)] \(f_i(A^\ast_i)\subseteq B^\ast_i\) for all \(i\geq 0\).  
\end{enumerate}
\end{defn}

\subsection{Augmented directed complexes with bases}
Augmented directed complexes are an abstraction of the theory of pasting diagrams, so we should certainly have the ability to construct them in terms of generating sets.
\begin{defn}
We say that a graded subset \(X_\bullet\subseteq A_\bullet\) of an augmented directed complex \(A\) is a \dfn{basis} for \(A\) if \(A_i\) is a free abelian group on \(X_i\), and if \(A^\ast_i\) is the submonoid of \(A_i\) consisting of the nonnegative linear combinations of the elements in \(X_i\).  
\end{defn}
If \(A\) is an augmented directed complex, we may equip each \(A_i\) with an order structure, where we say \[x \leq y \quad \text{if and only if} \quad y-x \in A^\ast_i.\]  
When \(A\) has a basis, this makes each \(A_i\) into a partially ordered abelian group.  Since the basis elements of a free commutative monoid are indecomposable, the basis elements in degree \(i\)  can be characterized as the minimal elements of \(A_i\) with respect to this order structure.  Moreover, \(A_i\) is a lattice, where any pair of elements has a least upper bound \(x\vee y\) and a greatest lower bound \(x\wedge y\).  In fact, in this situation, for any element \(x\in A_{i+1}\), there exist unique elements \(\partial^+(x)\) and \(\partial^-(x)\) in \(A^*_i\) such that \[\partial(x)=\partial^+(x) - \partial^-(x), \qquad \partial^+(x) \wedge \partial^-(x) = 0.\]

For any globular set \(X\), we may form an augmented directed complex with basis \(X\) as follows: Let \(C(X)=\mathbf{Z}[X]\) be the free graded abelian group on the graded set underlying \(X\), and let \(C^\ast(X)\mathbf{N}[X]\) be the free graded commutative submonoid of \(\mathbf{Z}[X]\) on the basis \(X\).    We define the boundary map \(\partial: C(X)_{i+1} \to C(X)_i\) to be the map defined by sending \(x\mapsto t(x)-s(x)\) for each \(x\in X_i\).  Finally, for each \(x\in X_0\), we let \(\varepsilon(x)=1\).  This determines a functor from globular sets to augmented directed complexes.  

\begin{thm} The full subcategory of \(\cat{ADC}\) spanned by the objects \(C(t)\) for \(t\) an object of \(\Theta_0\) is equivalent to \(\Theta\).  
\end{thm}
\begin{proof} This is one of the main theorems of \cite{steinersimple}.
\end{proof}

There is a tensor product on the category of augmented directed complexes inherited from the category of augmented chain complexes defined as follows:
\begin{enumerate}
\item[(i)] We have that \((A\otimes B)_n = \bigoplus_{i+j=n} A_i \otimes B_j\).  
\item[(ii)] The boundary is defined by sending \(a\otimes b \mapsto \partial_A(a)\otimes b + (-1)^{\heit(b)} a\otimes \partial_B(b)\)
\item[(iii)] The augmentation map \((A\otimes B)_0\to \mathbf{Z}\) is given by \(a\otimes b \mapsto \varepsilon_A(a)\varepsilon_B(b)\).
\item[(iv)] The submonoids \((A\otimes B)^\ast\) are defined to be the submonoids generated by the tensors \(a\otimes b\) such that \(a\in A^\ast\) and \(b\in B^\ast\) of appropriate heights.  
\end{enumerate}
\begin{thm} This tensor product extends to a tensor product on the category of all strict \(\omega\)-categories.
\end{thm}
\begin{proof} This is a theorem in \cite{steinerchain}.  
\end{proof}  