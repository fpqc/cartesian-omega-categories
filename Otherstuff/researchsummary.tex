\documentclass{amsart}
\usepackage{amssymb}
\begin{document}
\theoremstyle{plain}
\newtheorem*{thm}{Theorem}
\newtheorem*{lem}{Lemma}
\newtheorem*{cor}{Corollary}
\newtheorem*{prop}{Proposition}
\theoremstyle{definition}
\newtheorem{defn}{Definition}
\newtheorem{rmk}{Remark}
\title{Quasicategories and their mapping spaces: A literature Review}
\author{Harry Gindi}
\date{\today}
\address{University of Michigan\\ 530 Church Street\\ Ann Arbor, Michigan 48109}
\subjclass[2010]{91F20}
\begin{abstract}
The theory of quasi-categories [J1] (introduced originally as Weak Kan Complexes in [BV] and also called $\infty$-categories in [L]) is a geometric approach to higher category theory.  It has been shown by the collective work of a number of people that the theory of quasi-categories is Quillen-equivalent to both Rezk's model category of complete segal spaces, Bergner's model category of simplicial categories [Be], the model category of Segal categories of Dwyer-Kan-Smith, and Barwick-Kan's model category of relative categories [B1].  Collectively, we call these classes simplicial models of $(\infty,1)$-categories.  The theory of quasi-categories has been built up more recently by Lurie, who has used the theory as a foundation for his approach to the theory of derived algebraic geometry. His and Joyal's work have really made quasi-categories the most mature model for $(\infty,1)$-categories, with almost all of ordinary category theory replicated and generalized. We will recall some definitions and state, without proof, some of the important theorems, specificially those showing that the different definitions of mapping spaces are equivalent up to coherent  natural homotopy.
\end{abstract}
\maketitle
\section{Introduction}
Let $X$ be a topological space, and consider its fundamental groupoid $\Pi_1(X)$, which is given by the following data: Objects of $X$ are the elements of $X$, and morphisms $Hom_{\Pi_1(X)}(x,y)$ for points $x,y\in X$ are given by homotopy classes of paths $f:[0,1]\to X$ such that $f(0)=x$ and $f(1)=y$.  This construction gives a precise way to encode all of the data of all of the local dimension-1 homotopy data around any point $x$ as a single structure. That is, it is a way to look at the big picture all at once rather than by probing the darkness one point at a time. It's important to observe that $\Pi_1(X)$ also contains the data necessary to define $\Pi_0$, the set of connected components. 

There is a similar construction using strict $2$-categories to capture the local information of all of the second homotopy groups of $x$ at each point, where the objects of $\Pi_2(X)$ are defined to be the points of $X$, the morphism classes are defined to be the set of all paths between two points (not homotopy classes), and finally, the $2$-cells are precisely homotopy classes of homotopies between paths.  However, this construction actually contains even more data than we had from $\Pi_1$.  We can actually recover from it all of the data of the fundamental groupoid, the connected components, all of the second homotopy groups of $X$ at each point $x$, and also the homotopy-group actions between the groups of the space.  Unfortunately, things at this point become far more difficult.  It's well known that $\Pi_3(S_2)$  does not admit the structure of a strict $3$-category [L], and it's also quite clear that actually keeping track of all of the higher associativity and unit laws becomes very difficult very quickly.   For instance, the definition of the fundamental 4-groupoid has to keep track of some forty to fifty pages of coherence diagrams and morphisms.    

However, it's important to notice that the idea itself makes sense.  The homotopical n-data of a space \emph{morally} have the structure of an $n$-category, with the $n$-cells given by homotopies between $n-1$-cells, even though there are rather technical reasons why this idea isn't \emph{strictly} true.  However, since the notion of a path between points is in some sense obviously invertible (that is, if we run the same exact path backwards, we get the inverse path by concatenation), we can rescue the idea in the case that all morphisms are invertible by introducing \emph{diagram-filling conditions}.  That is, since direction does not matter, we can avoid having to specify higher data explicitly and instead give conditions that ``witness'' the commutativity of the important diagrams by demanding that such diagrams can be filled, and further, that two different fillers give the same result up to homotopy.  This is the philosophy behind the quasi-categorical approach to higher category theory.  

\section{Classical Homotopy theory}

There have been many attempts to generalize and weaken category theory since the mid-1960s.  Jean B\'enabou defined a \emph{bicategory} to be a weak $2$-category. That is, a $2$-category where associativity and unit are preserved under composition only up to coherent isomorphism.  The notion of a pseudofunctor, a structure preserving map of $2$-categories was developed during the S\'eminaire de G\'eom\'etrie Alg\'ebrique du bois Marie held by Alexander Grothendieck in the 1960s to develop abstract algebraic geometry and to keep track of more data than ordinary sheaf theory would allow.  Because bicategories and 2-categories have a whole extra dimension, they can be used to track local symmetries and save data that would have been discarded by just taking the quotient by an equivalence relation.  Unfortunately, while the axioms for a bicategory take up a scanty four pages, the axioms for a tricategory take up nearly 15, and the axioms for a tetracategory consist of nearly sixty pages of coherence diagrams and axioms.  This is what category theorists call the ``combinatorial explosion'' in higher category theory.  

Clearly there must be a better way to track all of that combinatorial data in some sort of mathematical structure.  Cue simplicial sets.

\begin{defn}
Let $\Delta$ denote the category of finite nonempty ordinals with order-preserving maps between them. We call this category the category of combinatorial simplices, or simply the \emph{simplex category}. A simplicial set is a functor $\Delta^{op}\to Set$, or a presheaf of sets on the simplex category.  A morphism of simplicial sets is merely a natural transformation.  This gives the class of simplicial sets the structure of a category, which we will denote by $sSet$.  Given a simplicial set $S$, we denote its value on the ordinal $[n]$ by $S_n$.  [H]
\end{defn}

We would like to say more, but we first must recall the following definitions:

\begin{defn}
A \emph{weak factorization system} on a category $X$ is a pair of (L,R) of classes of morphisms of $X$ such that $R=rlp(L)$, that is, $R$ is the class consisting of all morphisms of $X$ with the right lifting property (RLP) relative to all morphisms in $L$, and $L=llp(R)$, which means that $L$ is the class consisting of all morphisms of $X$ with the left lifting property (LLP) relative to all morphisms in $R$.  [J2]
\end{defn}

\begin{defn}
A Quillen model structure on a complete-cocomplete category $X$ is a triple $(C,W,F)$ of subsets of the class of arrows of $X$ satisfying the following axioms:
\begin{itemize}
\item The pairs $(C,W\cap F)$ and $(W\cap C, F)$ are weak factorization systems. 
\item $W$ satisfies the $2$-of-$3$ property and contains all isomorphisms.  This means that given any commutative triangle $a\to b\to c$ in $X$ such that if any two of the three edges $a\to b$, $b\to c$ or $a\to c$ is in $W$, then so is the third. (Note: The diagram $\cdot\to \cdot\to \cdot$ is standard notation for a commutative triangle, since the edge $a\to c$ is simply defined to be the composite of the maps $a\to b$ and $b\to c$). [Q1]
\end{itemize}
\end{defn}

\begin{rmk}
Given a Quillen model category $X$ with structure $(C,W,F)$, we call morphisms belonging to the class $C\cap W$ (respectively $W \cap F$) the class of \emph{trivial cofibrations} (resp. \emph{trivial fibrations}).  
\end{rmk}

\begin{rmk}
Sometimes we require further that a model category admit two \emph{functorial} factorizations of maps for the weak factorization systems.  This can be constructed afterwards if there exist small generating sets for the classes of trivial cofibrations and cofibrations.  This effectively uses Quillen's \emph{small object argument}, which is an incredibly powerful tool used to generate functorial factorizations.  See [Q1] or [H] for more details.
\end{rmk}

As is usually the case in mathematics, not only do we have to give a definition of the relevant objects, but also the relevant structure-preserving maps between them.  

\begin{defn}
A morphism $X\to Y$ between Quillen model categories is defined to be an adjunction $L:X\leftrightarrows Y:R$ such that $L$ preserves cofibrations and trivial cofibrations.  We say that a left Quillen functor $L:X\to Y$ is a \emph{left Quillen equivalence} if its total left derived functor (the right Kan extension of $\gamma_Y \circ F$ along $\gamma_X$)  is an equivalence of categories.  
\end{defn}

The theory of simplicial sets dates back to Eilenberg and Zilber, who discovered them while looking for purely combinatorial models of spaces for use in homological algebra.  The category of simplicial sets admits a Quillen model structure, which we will call the Kan model structure consisting of the following data:

\begin{itemize}
\item The class of cofibrations is the class of all monomorphisms of $sSet$.
\item The class of weak equivalences is the class of weak homotopy equivalences, that is, those maps which induce isomorphisms of all simplicial homotopy groups at every point and a bijection on connected components.
\item The class of fibrations consists of the Kan fibrations.  Recall that a morphism of simplicial sets is called a Kan fibration if it has the right lifting property (henceforth RLP) with respect to all horn inclusions. [Q1]
\end{itemize}

For the reader's sake, we recall the definition of a horn inclusion: By the Yoneda lemma, we have representable functors for each $[n]\in \Delta$.  We will denote them by $\Delta^n:=Hom(-,[n])$.  Recall further that given a simplicial set $S$, the Yoneda lemma implies that the set $S_n$ is in natural bijection with $Hom(\Delta^n,S)$.   Since the Yoneda embedding is full and faithful, we may describe the elements of $\Delta^n_m$ for any $m\in \mathbb{N}$ in terms of the set of morphisms living downstairs in the category $\Delta$.  That is, $\Delta^n_m\cong Hom(\Delta^m,\Delta^n)\cong Hom([m],[n])$.  We define a subfunctor:

\begin{defn} $\Lambda^n_i\subseteq \Delta^n$ called the $i^{th}$ horn of $\Delta^n$ for each $0\leq i\leq n$ as follows: Let $(\Lambda^n_i)_m=\{p:[m]\to [n]:p([m])\cup\{i\}\neq [n]\}$.  Geometrically, this corresponds to an $n$-simplex drained of its interior and missing the face opposite its $i^{th}$ vertex.  A horn inclusion is the natural inclusion map of the form $\Lambda^n_i\to \Delta^n$.  If $0<i<n$, we will call such a horn inclusion an \emph{inner} horn inclusion. [Q1] 
\end{defn}

By a beautiful theorem of Dan Quillen, the Kan Model structure on the category of simplicial sets is left-Quillen equivalent to the category of compactly-generated topological spaces with the Quillen model structure (defined by interval inclusions, weak homotopy equivalence, and Serre fibrations) by means of the geometric realization and the total singular complex adjunction $|\cdot|:SS\rightleftarrows CGTop: Sing$ [Q1].  However, these arguments rely on lifting the model structure from topological spaces to simplicial sets.  For a completely elementary argument independent of the category of topological spaces, see \emph{Les pr\'efaisceaux commes mod\`eles des types d'homotopie} by D. C. Cisinski (Ast\'erisque 308). 

\section{Quasi-categories}

Quasi-categories were discovered by Boardman and Vogt in the 1970s [BV].  They called them weak Kan complexes for the reason we will see in the following definition:

\begin{defn}
A simplicial set $S$ is called a quasi-category if the terminal map $S\to *$ has the right lifting property with respect to all \emph{inner} horn inclusions.
\end{defn}

That is, a quasi-category is almost a Kan complex, except we can't always fill in initial or terminal horns. What does this mean?  Take the following case for example:  Let $S$ be a quasi-category.  Given a horn $\Lambda^2_1\to S$ classifying two edges, we can lift this map to a map of a simplex.  That is, we can find a composite representing the concatenation of two edges.  That is, we can compose maps!  Further, it's not hard to check that composition is associative up to coherent homotopy.  We leave the computation as an exercise for the reader. However, where quasi-categories differ from Kan complexes is that we can't lift initial and terminal horns.  Suppose we have a horn map $\Lambda^2_2\to S$ classifying the identity on its bottom edge, and classifying some edge.  If we have a lift, this says that the right edge admits a left inverse.  Since Kan complexes have this property, however, it's easy to see that this implies that all edges of a Kan complex are equivalences.  In the sequel, Kan complexes will be our model for $(\infty,0)$-categories, or $\infty$-groupoids.   

Perhaps the first order of business is to give an explicit definition of the mapping spaces in a quasicategory.  There are a number of equivalent approaches (of course, only equivalent up to coherent homtopy), but we have not yet developed enough of the theory to give them all.  We will give the simplest definition first in terms of the comonadic simplicial resolution (this is often called the bar construction), of which we recall the definition below:

Let $M$ be a monad based at $E$, and let $E^M$ be its category of algebras.  This induces an adjunction $F:E\rightleftarrows E^M:U$ called the \emph{free-forgetful} adjunction.  Taking $W=FU:E^M\to E^M$, this gives a comonad $W$ based at $E^M$ with comultiplication $\psi: W\to W\circ W$ and counit $\phi:Id\to W$.  We define a sequence of functors by recursion: $W^n=W\circ W^{n-1}$ and $W^0=Id_{E^M}$.  We define a functor $\bar{W}:\Delta^{op}\to [E^M,E^M]$ sending $[n]\mapsto W^{n+1}$.  On morphisms, we send faces $d_i:[k]\to[k-1]\mapsto W^i\phi W^{k-i}:W^{k+1}\to {W^k}$, and for degeneracies, we send $s_i:[k-1]\to [k]\mapsto W^i\psi W^{k-i}:W^k\to W^{k+1}$.  We let $\bar{W}_n=\bar{W}([n])$, and this has another argument for on object of $E^m$ by currying arguments.  Indeed, it is immediate that given an object $X$ of $E^M$, $\bar{W}X$ is a simplicial object in $E^M$ [DS].  

Let $Q$ denote the category of quivers (also called digraphs or diagram schemes).  Recall that this category is equipped with a monad, called the  $Cat$-monad.  The category of algebras for this monad is, as one would guess, $Cat$, the category of categories (suppressing issues of size).  We call the corresponding comonad $F$ on $Cat$ the \emph{free category} comonad.  Using the bar construction from the previous paragraph, this gives us a resolution for any ordinary category $C$.  We may identify $\Delta$ with a full subcategory of $Cat$.  Applying the bar construction to $F$, and restricting $\bar{F}$ to  $\Delta$, this gives a functor $\bar{F}:\Delta\to sCat$.  By abstract nonsense, we can extend the functor $\bar{F}$ to a colimit preserving functor $\hat{F}:Set^{\Delta^{op}}\to sCat$.  To see this, suppose we are given a simplicial set $S$, it is a colimit of its simplices by Yoneda's lemma, we define $\hat{F}(S)=colim_{\Delta\downarrow S}\bar{F}([n])$ preserves colimits, this gives a functor $sSet\to sCat$. [DS] We will drop the notation $\bar{F}$ and replace it with $\mathfrak{C}$ in the sequel, since this is more standard notation. Finally, we can define the mapping space of two vertices in a simplicial set:

\begin{defn}
Let $S$ be a simplicial set, and let $x,y\in S_0$ be two vertices of $S$.  Then we define the mapping space object $Map_S(x,y)$ to be the honest mapping space object $Map_{\mathfrak{C}(S)}(x,y)$ in the associated simplicial category [Be].  
\end{defn}

To state the other definitions, we must define the slice quasicategory construction.  To do this, we notice that $\Delta$ admits a strict monoidal structure given by the ordinal sum of two ordinals $[m]+[n]=[m+n+1]$.  By abstract nonsense, the \emph{Day Convolution} gives an extension of this functor to any simplicial sets.  We call this operation the \emph{join}, and it is denoted by $X\star Y$.  Fixing each variable, we can see that the join can also be seen to give a unique canonical map $Y=Y\star\emptyset \to Y\star X$, which means that the functor can be sharpened to a functor $Y\star -:sSet\to Y\downarrow sSet$.  This functor commutes with all colimits, and therefore, it has a right adjoint.  We call this new functor the simplicial underslice.  It is denoted by $S_{p/}$ where $p:Y\to S$ is an object of $Y\downarrow sSet$. If we instead filled $Y$ into the second coordinate, this would give us a functor called the simplicial overslice, denoted by $S_{/p}$.  [J2] 

Given a simplicial set $S$, and given two points $x,y\in S_0$, we can view these vertices as classifying maps $x:\Delta^0\to S$ and $y:\Delta^{0}\to S$.  We can define the space of \emph{right} morphisms from $x$ to $y$ in $S$ to be the simplicial set $Hom^R_S(x,y):\{x\}\times_S S_{/y}$.  Dually, we may define the space of left morphisms by taking $Hom^L_S(x,y)=S_{x/}\times_S \{y\}$.  It turns out that these descriptions can be quite useful depending on the problem at hand.  It is a theorem of Lurie that these (including the earlier bar construction definition) are equivalent  to one another up to coherent homotopy whenever $S$ is a quasi-category. [L] 

\section{Results}

There are numerous interesting results about quasi-categories, which generalize the classical theorems of category theory and homotopy theory.  We will cite some below: 

\begin{thm}[Dwyer-Kan]
Let $(X,W)$ be a relative category, that is, a category $X$ equipped with a subcategory $W$.  Then we can form the simplicial localization $L_WX$ of $X$ at the class $W$ such that the simplicial homotopy category of $L_WX$ is equivalent to the ordinary localization (that is, the image of $L_WX$ under the base change functor $\pi_0: sSet\to Set$ is canonically equivalent to the ordinary localization $W^{-1}X$).  Further, this localization is functorial with respect to $W$ and $X$.  [DK]
\end{thm}

This says in particular that we can boost up all of the localization data to a simplicial enrichment that characterizes the homotopical data of the pair $(X,W)$.  This is particularly nice, since it means that we can equip the pair $(X,W)$ with a structure that allows us to perform homotopical operations on it, in particular, homotopy limits, homotopy colimits, and homotopy Kan extensions.  We may combine this with a theorem of Barwick and Kan to get the following even stronger result:

\begin{thm}[Barwick-Kan]
The category of relative categories is Quillen equivalent to the category of simplicially enriched categories. [B1] [B2]
\end{thm}

In particular, this says that any simplicially enriched category can be recovered up to homotopy as the simplicial localization of a relative category.  As in other areas of mathematics, we call such data a \emph{presentation}.  We can further extend this result by applying the following theorem:

\begin{thm}[Lurie]
Consider the Joyal model structure specified on $sSet$ as the model structure for which the cofibrations are exactly the monomorphisms and the fibrant objects are quasicategories.  This model category is Quillen equivalent to the category of simplicial categories.  [L]
\end{thm}

from which it follows that

\begin{thm}[Joyal]
Every quasi-category admits a presentation as the classifying space of some ordinary category localized at some subcategory.  In particular, the simplicial localization can be described on quasi-categories in a very simple way: Given some morphisms $W$ to be localized, we can take the pushout of the classifying space of $X$ over the maps classifying edges belonging to $W$ by the inclusion $\coprod_W\Delta^1\to \coprod_WJ$ where $J$ is the classifying space of the unique groupoid with two objects and two non-identity edges. [J2]
\end{thm}

Because the mapping spaces noted above are hard to work with in general, since they tend to be so big, Spivak and Dugger have given a very powerful presentation for them in terms of objects called necklaces.  We require some more definitions before proceeding to state the actual theorem.

\begin{defn}
A combinatorial necklace is a finite length initial-to-terminal wedge of variable dimension $n$-simplices, that is, a combinatorial necklace is an object of $sSet$ of the form: $$\Delta^{n_0}\coprod_{\Delta^0}\Delta^{n_1}\coprod_{\Delta^0}....\coprod_{\Delta^0}\Delta^{n_{k-1}}\coprod_{\Delta^0} \Delta^{n_k}$$ where the $\Delta^0$ terms are mapping in by $n_i$ on the left and $0$ on the right for all $i$.  We say that a combinatorial necklace is \emph{reduced} if $n_i\neq 0$ for any $i$.  [DS]
\end{defn}

We notice that any combinatorial necklace $T$ admits a unique map $\partial \Delta^1\to T$ sending $0$ to the initial vertex of the chain and $1$ to the final vertex of the chain.  We identify the category of all combinatorial necklaces with the full subcategory of $\partial\Delta^1\downarrow sSet$ spanned by the necklaces relative the unique map $\partial \Delta^1\to T$.  Morally, this says that morphisms of necklaces are those morphisms of simplicial sets that preserve the bipointing induced by the canonical map.  We call this category $Nec$.  A map $T\to S$ is referred to as a necklace \emph{in} $X$.  We say that a necklace $T\to S$ is \emph{between} $x,y\in S$ if its initial vertex maps to $x$ and its terminal vertex maps to $y$.  Define the set of all necklaces between $x$ and $y$ to just be the set $Nec_S(x,y)$.  We equip $Nec_S(x,y)$ with the structure of a category by defining its maps to be commutative diagrams $T\to T'\to S$ preserving the initial and terminal vertices.


\begin{thm}[Dugger-Spivak]
Given two vertices $x,y\in S_0$ for a simplicial set $S$, the simplicial set $N(Nec_{S}(x,y))$ can be connected by a natural zig-zag of weak equivalences to the simplicial set $Map_S(x,y)$. [DS]
\end{thm}
  
It's reasonable to ask why we need so many equivalent models for function spaces.  The answer is quite interesting and quite subtle.  The short answer is that each serves its purpose when necessary.  Consider, for example, the space of right morphisms between two points in a quasicategory.  In the process of proving that there is a Quillen equivalence between the simplicial model category $sSet\downarrow X$ (with its contravariant model structure) and the simplicial model category of simplicial functors $sSet^{\mathfrak{C}(X)}$ with the projective model structure, Lurie introduces the straightening and unstraightening functors.  Interestingly enough, it turns out that $Hom^R_{N(\mathfrak{C}(X))}(x,y)$ is isomorphic on-the-nose to the unstraightening over a point $z$ of $Map_{\mathfrak{C}(S)}(x,y)$.  That is, the space of right homomorphisms naturally comes up in the proof of this very important fact.

Dugger-Spivak give a more concise and substantially less intricate proof of the the fact that $\mathfrak{C}$ is a left quillen equivalence in [DS] using their necklace formalism as well.  The main interesting feature of their construction is that their morphism spaces are always nerves of categories.  In particular, the nerve of a category has a very simple structure, since it is $1$-\emph{truncated} and generated by the underlying 1-categorical data. The standard construction of Dwyer-Kan or Lurie turns this into a very nasty object, since it forces us to take colimits of the comonad resolutions of ordinary categories.  This is an extremely difficult thing to do with simplicially enriched categories, which is at least as hard as taking colimits of ordinary categories, which is is itself a very difficult problem because colimits of essentially algebraic theories tend to be given as quotients of free algebraic objects by free subobjects. To illustrate this point, given a free group $G$ on finitely many generators and a normal free subgroup on finitely many generators $H$, there is no decidable algorithm to determine whether or not the quotient group is trivial.  Obviously, when the sets involved are infinite, it becomes essentially impossible to derive any sort of real information from such a presentation.  In the  case of free categories or free simplicial categories, the problem becomes that much harder (since groups can just be viewed as categories with a single object! However, colimits in $sSet\downarrow\mathfrak{C}(X)$ can simply be described as object-wise colimits, so it is substantially easier to construct Joyal's simplicial localization than a colimit of simplicial categories.    Using Dugger-Spivak's results, we can completely eliminate the need for any sort of colimit in the formula, which makes things quite a bit more computable (although at this level of generality, it's reasonable to expect that computation will still be extremely difficult anyway).  

%%%%%%%%%%%%%%%%%%%%%%%%%%%%%%%%%%%%%%%%%%%%%%%%%%%%%%%%%%%%%%%

\bibliographystyle{amsalpha}
\begin{thebibliography}{JTT}

\bibitem[Be]{Bergner} J. Bergner, {\em A model category structure on
the category of simplicial categories\/}, Trans. Amer. Math. Soc. {\bf
359} (2007), no. 5, 2043--2058.

\bibitem[B1]{BK1} Barwick, C. and D.M. Kan. {\it Relative categories as another model for the homotopy theory of homotopy theories, I. The model structure.} preprint, 2009.

\bibitem[B2]{BK2} Barwick, C. and D.M. Kan. {\it Relative categories as another model for the homotopy theory of homotopy theories, II. The weak equivalences.} preprint, 2009.

\bibitem[BV]{BV} Boardman, J.M. and R.M. Vogt. {\em Homotopy Invariant Structures on Topological Spaces\/} Lecture Notes in Mathematics 347. Springer-Verlag, Berlin and New York, 1973.

\bibitem[DK]{DK} Dwyer, W.G. and D.M. Kan. {\it Simplicial localizations
of categories.} J. Pure Appl. Algebra 17, (1980), 267-284.

\bibitem[DH]{DHKS} Dwyer, W.G., Hirschhorn, P.S., Kan, D. and J. Smith. {\em Homotopy Limit Functors on Model Categories and Homotopical Categories\/}, Mathematical surveys and monographs 113 (2004).

\bibitem[DS]{DS} D. Dugger and D. Spivak, {\em Mapping spaces in
quasi-categories\/}, preprint, 2009.

\bibitem[H]{H} P.~J. Hirschhorn, {\em Model categories and
localizations\/}, Mathematical Surveys and Monographs {\bf 99},
American Mathematical Society, Providence, RI, 2003.

\bibitem[J1]{J1} A. Joyal, {\em Quasi-categories and Kan complexes\/},
J. Pure Appl. Algebra {\bf 175} (2002), no. 1--3, 207--222.

\bibitem[J2]{J2} A. Joyal, {\em The theory of quasi-categories\/}, preprint.

\bibitem[L]{L} J. Lurie, {\em Higher topos theory\/}, Annals of
Mathematics Studies {\bf 170}, Princeton University Press, Princeton,
NJ, 2009.

\bibitem[Q1]{Q1} D. Quillen {\em Homotopical Algebra\/}, Lectures Notes in Mathematics 43, Springer-Verlag, Berlin, 1967. 

\bibitem[Q2]{Q} D. Quillen, {\em Higher algebraic $K$-theory. I.\/},
Algebraic $K$-theory, I: Higher $K$-theories (Proc. Conf., Batelle
Memorial Inst., Seattle, Wash., 1972), pp. 85--147.  Lecture Notes in
Math. {\bf 341}, Springer, Berlin, 1973.    


\end{thebibliography}
\end{document}