\begin{defn} We say that a map of cellular sets \(f:X\to S\) is an \dfn{inner fibration} if it has the right lifting property with respect to all smash products of spine inclusions with monomorphisms.  That is, if it has the right lifting property with respect to all maps of the form \[\Sp[t] \times B \cup \Theta[t] \times A\hookrightarrow \Theta[t]\times B\] where \(A\hookrightarrow B\) is a monomorphism and \([t]\) is an object of \(\Theta\).  

If \(X\) is a cellular set such that the terminal map \(X\to \Theta[0]\) is an inner fibration, we say that \(X\) is \dfn{inner fibrant}. 
\end{defn}


\begin{prop}A morphism \(X\to \mathfrak{N}(Y)\), where \(Y\) is a strict \(\omega\)-category is an inner fibration if and only if \(X\) is inner-fibrant.
\end{prop}
\begin{proof}
It suffices to show that the induced map \[X^{\Theta[t]}\to \mathfrak{N}\mathcal{P}(X)^{\Theta[t]}\times_{\mathfrak{N}\mathcal{P}(X)^{\Sp[t]}} X^{\Sp[t]}\] is a trivial fibration for every object \([t]\) of \(\Theta\) if and only if \(X\) is inner-fibrant.  However, since \(\mathfrak{N}(Y)\) is the nerve of a strict \(\omega\)-category, the map \[\mathfrak{N}(X)^{\Theta[t]}\to \mathfrak{N}(Y)^{\Sp[t]}\] is an isomorphism, so it suffices to show that the map \[X^{\Theta[t]}\to X^{\Sp[t]}\] is a trivial fibration if and only if \(X\) is inner-fibrant, but this is precisely the definition.  
\end{proof}

\subsection{The strict homotopy \(\omega\)-category}\label{hocat}

Given a cellular set \(X\), we may form a strict \(\omega\)-category \(\mathcal{P}(X)\), where \(\mathcal{P}\) denotes the left adjoint of the cellular nerve.  We will describe this strict \(\omega\)-category in terms of generators and relations as follows:
We define a set of maps \(I=\{\partial D_n \hookrightarrow D_n: n\geq 0\}\) in the category of strict \(\omega\)-categories such that we can construct a labeled \(I\)-CW complex \(\mathcal{P}_0(X)\), from which we can obtain \(\mathcal{P}(X)\) by taking quotients by appropriate relations. We define \(\mathcal{P}^{(0)}_0(X)\) to be the set of all vertices in \(X\).  Then for \(n>0\), we define \(\mathcal{P}^{(n)}_0(X)\) by a pushout diagram:
\begin{equation*}
\begin{tikzpicture}
\matrix (b) [matrix of math nodes, row sep=3em,
column sep=3em, text height=1.5ex, text depth=0.25ex]
{ C_n\times \partial D_n & \mathcal{P}^{(n-1)}_0(X) \\
   C_n \times D_n &   \mathcal{P}^{(n)}_0(X)\\};
\path[->, font=\scriptsize]
(b-1-1) edge (b-1-2)
        edge (b-2-1)
(b-2-1) edge (b-2-2)
(b-1-2) edge (b-2-2);
\end{tikzpicture},
\end{equation*}
where \(C_n\) is defined to be the set of all \(n\)-cells \(D_n\to X\).  The maps \(C_n\times \partial D_n\to \mathcal{P}^{(n-1)}_0(X)\) classifying the boundary of the relevant \(n\)-cell.  The map \(C_n\times \partial D_n \to C_n\times D_n\) is just given by the disjoint union \(C_n \times \iota_n=\coprod_{c\in C_n} \iota_n,\) where \(\iota_n:\partial D_n\hookrightarrow D_n\) is the usual inclusion of the boundary.  Then we obtain \(\mathcal{P}_0(X)\) as the the inductive limit of the system
 \begin{equation*}
\begin{tikzpicture}
\matrix (a) [matrix of math nodes, row sep=3em,
column sep=3em, text height=1.5ex, text depth=0.25ex]
{ \mathcal{P}^{(0)}_0(X) & \dots & \mathcal{P}^{(n)}_0(X) & \dots \\};
\path[right hook->, font=\scriptsize]
(a-1-1) edge (a-1-2)
(a-1-2) edge (a-1-3)
(a-1-3) edge (a-1-4);
\end{tikzpicture}.
\end{equation*}
We define the labeling of \(\mathcal{P}_0(X)\) to be the induced map \(\coprod_{n\geq 0} C_n \times D_n \to \mathcal{P}_0(X)\).  We notice that every cell is a formal finite composite of labeled cells.  An ordered family of labeled cells \(c_1,\dots,c_n\) is called \dfn{\([t]\)-composable} if there exists a map \(c:\Sp[t]\to X\) of width \(n\) whose restriction to its \(i\)th input sector is \(c_i\) for  all \( 1 \leq i\leq n\).  We denote the formal \(\Sp[t]\)-shaped composite of such family of cells by \(\operatorname{comp}_t(c_1,\dots,c_n)\), which we will also denote by \(\operatorname{comp}_t(c)\) whenever such a family is just defined to be the ordered family of input sectors of a map \(\Sp[t]\to X\).     

The defining relations on \(\mathcal{P}_0(X)\) arise canonically from the elements \(\Theta[t]\to X,\) as follows: Given any map \(g:\Theta[t]\to X\), we obtain by restriction to the spine (resp. cospine) a map \(c_g:\Sp[t]\to X\) (resp. \(d_g:\coSp[t]\to X\)).  The spine classifies a \([t]\)-composable family of labeled cells in \(\mathcal{P}_0\) whose composite therein is the formal composite \(\operatorname{comp}_t(c_g)\).  Since the map \(X\to \mathfrak{N}\mathcal{P}(X)\) must also be defined on each object \(\Theta[t]\to X\), and since any composable family of cells in \(\mathcal{P}(X)\) must have a unique composite, it follows that we must take the quotient of \(\mathcal{P}_0\) by the equivalence relation generated by the relation on \(\mathcal{P}_0(X)\), 
where if \(g:\Theta[t]\to X\) is a map, then \(\operatorname{comp}_t c_g \sim d_g\). 

It is not hard to see that the strict \(\omega\)-category  obtained by taking the quotient of \(\mathcal{P}_0(X)\) by the equivalence relation defined above is precisely \(\mathcal{P}(X)\).  This follows from the fact that it is obtained first by taking a formal closure of \(X\) under composition, then obtained by identifying all non-unique composites, both of which are initial procedures for their intended purposes.  

\begin{prop} If \(X\) is a \(\W_{\Sp}\)-fibrant cellular set, then the corresponding component of the unit map \(\eta_X:X\to \mathfrak{N}\mathcal{P}(X)\) is a surjective \(\W_{\Sp}\) fibration. 
\end{prop}
\begin{proof} Every globe in \(\mathcal{P}(X)\) is represented by a formal composite of labeled disks in \(\mathcal{P}_0(X)\).  Then we may choose a spine \(\Sp[t]\to X\) representing that formal composite.  Since \(X\) is inner-fibrant, there exists an extension \(\Theta[t]\to X\).  This means that in \(\mathcal{P}(X)\), the formal composite is identified with the cospine of that extension.  Therefore, every globe in \(\mathcal{P}(X)\) is in the image of the unit morphism.  

Since \(X\) is \(\W_{\Sp}\)-fibrant and \(\mathcal{P}(X)\) is a strict \(\omega\)-category, the unit map is an inner fibration, so it will suffice to show that it satisfies the right lifting property with respect to the map \(j:e\to J\).  However, this follows from the fact that the homotopy category of \(\mathfrak{M}(e,X)\) is isomorphic to the \(1\)-truncation of the strict \(\omega\)-category \(\mathcal{P}(X)\).  
\end{proof}