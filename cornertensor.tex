\chapter{Corner Tensors and $\Theta$-horns}
We begin by recalling the theory of corner tensors developed in \cite{ourythesis}, which we will used to construct the inner horns inclusions of \(\Theta\).
\section{Corner tensors}
The corner tensor construction was introduced in \cite{ourythesis} as a generalization of the pushout product.  We specialize immediately to the case of tensors for arrows.  For the general case, see \cite{ourythesis}.

\subsection{Corner tensors and the cellularity lemma}
Let \(\C_0,\dots,\C_n,\D\) be a family of categories together with a functor \[\Box:\prod_{i=0}^n \C_i \to \D . \] Assume for the sake of argument that these categories are sufficiently cocomplete for what follows.

\begin{note} We have indexed things this way because we will be using it later in a setting where the argument with the smallest index plays a different role than the other arguments.  We will call the first argument the \dfn{\(0\)th argument} and reindex such that we have arguments numbered \(0,\dots,n\).   
\end{note}

Let \([1]\) be the free category on the linear quiver \([0\to 1]\).  This category can be endowed canonically with a monoidal product \(\wedge:[1]\times [1]\to [1]\) called the infimum, which is defined in the obvious way.   

\begin{defn} Using the Day convolution, the monoidal product on \([1]\) induces a functor \[\btensor:\prod_{i=0}^n \C^{[1]}_i\to \D\] given in coend form by the formula \[\btensor(f_0,\dots, f_n)_{\eta}\int^{\ell_0,\dots,\ell_n \in [1]} \overline{\Box}(f_0 \ell_0,\dots, f_n\ell_n)\times \Hom(\ell_0\wedge \dots \wedge \ell_n,\eta).\] We call this functor the \dfn{lower-corner tensor} of \(\Box\).  
\end{defn}

We will quickly state some elementary results, the proofs of which can be found in \cite{ourythesis}:
\begin{prop} If the \(kth\) argument of \(\Box\) preserves colimits of diagrams of a particular shape, then so too does \(\btensor\).  
\end{prop}

\begin{prop} Suppose we are given bifunctors \(\nabla_i:C_i \times C_i \to C_i\) and a bifunctor \(\nabla:D\times D\to D\) such that we have a natural multi-interchange isomorphism \[\Box(c_0,\dots,c_n)\nabla \Box(c^\prime_0,\dots, c^\prime_n) \cong \Box(c_0\nabla_0 c^\prime_0,\dots,c_n\nabla_n c^\prime_n).\]  Then if \(\Box\) and \(\nabla\) preserve colimits, we obtain a natural multi-interchange isomorphism \[\btensor(c_0,\dots,c_n)\overline{\nabla} \btensor(f^\prime_0,\dots, f^\prime_n) \cong \btensor(f_0\overline{\nabla}_0 f^\prime_0,\dots,f_n\overline{\nabla_n} f^\prime_n).\]
\end{prop}

Based on the proposition, we now take it as given that \(\Box\) preserves colimits.

\begin{obs}Given a family of arrows \(f=(f_0,\dots,f_n)\) with \(f_i:A_i\to B_i\) in \(\C_i\) for \(0\leq i\leq n\), the map \(\btensor(f_0,\dots,f_n)\) is exactly by the natural map \(P_f\to \Box(B_0,\dots,B_n)\), where \(P_f\) is the colimit of the diagram over \(\Box(B_0,\dots,B_n)\) defined by the functor \[\Box(f_0(\bullet),\dots, f_n(\bullet)) :E\to \overcat{\D}{\Box(B_0,\dots,B_n)}\] where \(E\) is the category defined by the poset \([1]^{n+1}-(1,\dots,1)\).  

If we let \(\widehat{f}_i=(f_0,\dots,\id_{\C_i},\dots, f_n)\), then the subdiagram on the maps \(\Box \widehat{f}_i\) for \(0\leq i\leq n\) is cofinal, which means that we may express \(P_f\) as its colimit. If we take the colimit \(Q\) of the subdiagram on the \(\Box \widehat{f}_i\) for \(0<i\leq n\), we obtain a map \(q:\Box(c_0,\dots, c_n\to Q\), and we may express \(P_f\) as the pushout \[\varinjlim(\Box(A_0, B_1, \dots, B_n)\xleftarrow{\Box \widehat{f}_0} \Box(A_0, A_1, \dots, A_n) \xrightarrow{q} Q) .\]  
\end{obs}

Fix a family of arrows \(g=(g_1,\dots, g_n)\) with \(g_i:X_i\to Y_i in \C_i\) for \(1\leq i \leq n\), and let \(f_0^\alpha:A^0_0\to A_0^1 ,f_0^\beta:B_0^0\to B_0^1\) be a pair of maps in \(\C_0\).  Fix a commutative square \(h:f_0^\alpha\to f_0^\beta\).  By the functoriality of \(\btensor\), we obtain a commutative square \(\btensor(h,g):\btensor(f_0^\alpha,g)\to \btensor(f_0^\beta,g)\), where \(\btensor(h,g)^0:P_{f_0^\alpha,g}\to P_{f_0^\beta,g}\) is the map induced by the functoriality of \(\btensor\).  We also see that \(\btensor(h,g)^1=\Box(h^1,\id_Y):\Box(A_0^1,Y) \to \Box(B_0^1,Y)\).  

\begin{prop}\label{idcubes} Given a family of maps \(g=(g_1,\dots, g_n)\) with \(g_i:A_i\to B_i in \C_i\) for \(1\leq i \leq n\) and a map \(f_0=\id_{X_0}:X_0\to X_0=Y_0\), then the map \(\btensor(f_0,g):P_{f_0,g}\to \Box(Y_0,B)\) is the identity map. 
\end{prop}
\begin{proof} Express \(P_{f_0,g}\) as the pushout of \(Q\) and \(\Box\widehat{f}_0\) as above.  Since \(\Box(\widehat{f}_0)=\Box(f_0,g)\), we see that \[\Box(\id_{X_0},\id_{B_1},\dots,g_i,\dots,\id_{B_n})\circ \Box(\widehat{g}_i)=\Box(f_0,g).\] By the construction of \(Q\), these glue together to a unique diagonal lift 
\begin{equation*}
\begin{tikzpicture}
\matrix (b) [matrix of math nodes, row sep=3em,
column sep=3em, text height=1.5ex, text depth=0.25ex]
{ \Box(X_0,A) & \Box(X_0,B) \\
Q & P_{f_0,g} & \Box(Y_0,B) \\};
\path[->, font=\scriptsize]
(b-1-1) edge node[auto]{\(\scriptstyle \Box(\widehat{f}_0)\)} (b-1-2)
        edge node[auto]{\(\scriptstyle q\)} (b-2-1)
(b-2-1) edge  (b-2-2)
				edge[dashed] (b-1-2)
(b-1-2) edge (b-2-2)
(b-2-2) edge[dashed] node[auto]{\(\scriptstyle \btensor(f_0,g)\)} (b-2-3);
\end{tikzpicture},
\end{equation*}   
but we also know that \(P_{f_0,g}\) is the pushout of this diagram, but the commutativity of the upper triangle implies that the pushout is \(\Box(X_0,B)\) itself.  Lastly, since the map \(\Box(X_0,B)\to \Box(Y_0,B)\) is the identity map and \(\Box(X_0,B)\) is \(Q_{f_0,g}\), it follows that the universal map \(\btensor(f_0,g)\) must be the identity map.
\end{proof}
\begin{prop} Fix a family of maps \(g=(g_1,\dots, g_n)\) with \(g_i:X_i\to Y_i\) in \(\C_i\) for \(1\leq i \leq n\), a pair of maps \(f_0^\alpha:A_0^0\to A_0^1\), \(f_0^\beta:B_0^0\to B_0^1\) in \(\C_0\).  If \(h\) be a \emph{cocartesian} square \(f_0^\alpha \to f_0^\beta\), then the square \(\btensor(h,g):\btensor(f_0^\alpha,g)\to \btensor(f_0^\beta,g)\) is cocartesian as well.  
\end{prop}
\begin{proof} Since \(\Box\) preserves pushouts argument by argument, it follows that the square \[\Box(h,\id_{B}):\Box(f_0^\alpha,\id_Y) \to \Box(f_0^\beta,\id_Y)\] is cocartesian.  Next, look at the square
\begin{equation*}
\begin{tikzpicture}
\matrix (b) [matrix of math nodes, row sep=3em,
column sep=3em, text height=1.5ex, text depth=0.25ex]
{ \Box(A_0^0,Y) & P_{f_0^\alpha,g} \\
  \Box(B_0^0,Y) & P_{f_0^\beta,g} \\};
\path[->, font=\scriptsize]
(b-1-1) edge (b-1-2)
(b-1-1)	edge node[auto]{\(\scriptstyle \Box(h^0,\id_Y)\)} (b-2-1)
(b-2-1) edge  (b-2-2)
(b-1-2) edge (b-2-2) ;
\end{tikzpicture} . 
\end{equation*}  
We can see that this square is cocartesian as follows: Since \(P_{f_0^\varepsilon,g}\) (for \(\epsilon \in \{\alpha,\beta\}\) decomposes as discussed earlier, we may look at 
\begin{align*}
\Box(B,Y) \coprod_{\Box(A,Y)} P_{f_0^\alpha,g}&=\Box(B,Y) \coprod_{\Box(A,Y)} \Box(A,Y)\coprod_{\Box(A,X)} Q_{f_0^\alpha,g}\\ 
&= \Box(B,Y) \coprod_{\Box(A,X)} Q_{f_0^\alpha,g}\\ 
&=\Box(B,Y) \coprod_{\Box(A,X)} \varinjlim F_1^n,
\intertext{where \(F_1^n\) denotes the subdiagram of the diagram defining \(P_{f_0^\alpha,g}\) spanned by the maps \(\Box(\widehat{g}_i)\) for \(1\leq i \leq n\)}
&=\Box(B,Y) \coprod_{\Box(B,X)}\varinjlim ( \Box(B,X) \coprod_{\Box(A,X)} F_1^n)\\
\intertext{It is easy to see from the explicit description of \(F_1^n\) in terms of the maps \(\Box(\widehat{g}_i)\) that the rightmost colimit above simplifies to \(Q_{f_0^\beta,g}\), since it is just the obvious cobase change and the functor \(\Box\), being compatible with colimits in each argument, is stable under cobase change.}
&=\Box(B,Y) \coprod_{\Box(B,X)} Q_{f_0^\beta,g}\\
&=P_{f_0^\beta,g}.
\end{align*}
Therefore, this square is also cocartesian.

Looking at the square\[\btensor(h,g):\btensor(f_0^\alpha,g)\to \btensor(f_0^\beta,g)\],  
\begin{equation*}
\begin{tikzpicture}
\matrix (b) [matrix of math nodes, row sep=3em,
column sep=3em, text height=1.5ex, text depth=0.25ex]
{  P_{f_0^\alpha,g} & \Box(A_0^1,Y)\\
   P_{f_0^\beta,g}  & \Box(B_0^1,Y)\\};
\path[->, font=\scriptsize]
(b-1-1) edge node[auto]{\(\scriptstyle \btensor(f_0^\alpha,g)\)}(b-1-2)
(b-1-1)	edge (b-2-1)
(b-2-1) edge node[auto]{\(\scriptstyle \btensor(f_0^\beta,g)\)}(b-2-2)
(b-1-2) edge node[auto]{\(\scriptstyle \Box(h^1,\id_Y)\)}(b-2-2) ;
\end{tikzpicture},
\end{equation*}  
we see that this square can be composed along the functorially-defined edge \(P_{f_0^\alpha,g}\to P_{f_0^\beta,g} \) with the previous square that we considered to form the first square that we considered in this proof.  Since we proved that the other two squares are cocartesian, it follows that this square is cocartesian, and therefore that \(f_0^\beta\) is a pushout of \(f_0^\alpha\).   
\end{proof}

\begin{obs} Given a pair of commutative squares
\begin{equation*}
\begin{tikzpicture}
\matrix (b) [matrix of math nodes, row sep=3em,
column sep=3em, text height=1.5ex, text depth=0.25ex]
{ a & b \\
  c & d \\};
\path[->, font=\scriptsize]
(b-1-1) edge node[auto]{\(\scriptstyle f\)}(b-1-2)
(b-1-1)	edge node[auto]{\(\scriptstyle l\)} (b-2-1)
(b-2-1) edge node[auto]{\(\scriptstyle g\)}(b-2-2)
(b-1-2) edge node[auto]{\(\scriptstyle r\)}(b-2-2);
\end{tikzpicture}
\qquad \qquad\qquad
\begin{tikzpicture}
\matrix (b) [matrix of math nodes, row sep=3em,
column sep=3em, text height=1.5ex, text depth=0.25ex]
{ c &  d  \\
  s &  s  \\};
\path[->, font=\scriptsize]
(b-1-1) edge node[auto]{\(\scriptstyle g\) } (b-1-2)
(b-1-1)	edge node[auto]{\(\scriptstyle \lambda\)} (b-2-1)
(b-2-1) edge node[auto]{\(\scriptstyle \id_S\)} (b-2-2)
(b-1-2) edge node[auto]{\(\scriptstyle \rho\)} (b-2-2) ;
\end{tikzpicture}, 
\end{equation*}  
in \(\C_0\) where the lefthand square is cocartesian, then the cube
\begin{equation*}
\begin{tikzpicture}
\matrix (b) [matrix of math nodes, row sep=3em,
column sep=3em, text height=1.5ex, text depth=0.25ex]
{ f &  \id_{b}  \\
  \lambda &  \rho  \\};
\path[->, font=\scriptsize]
(b-1-1) edge node[auto]{\(\scriptstyle (f,\id_b)\) } (b-1-2)
(b-1-1)	edge node[auto]{\(\scriptstyle (l,\lambda \circ l)\)} (b-2-1)
(b-2-1) edge node[auto]{\(\scriptstyle (r,\rho\circ r\)} (b-2-2)
(b-1-2) edge node[auto]{\(\scriptstyle (g,\id_s)\)} (b-2-2) ;
\end{tikzpicture}, 
\end{equation*}  
is a pushout in \(\C_0^{[1]}\) by merit of the fact that \(\lambda \circ l = \rho \circ r\).  
For any family \(h=(h_1,\dots,h_n)\) of morphisms with \(h_i:X_i\to Y_i\in \C_i\), applying \(\btensor(\bullet,h)\) to this square results in a further cocartesian square in \(\D^{[1]}\), 
\begin{equation*}
\begin{tikzpicture}
\matrix (b) [matrix of math nodes, row sep=6em,
column sep=5em, text height=1.5ex, text depth=0.25ex]
{ \btensor(f,h) &  \btensor(\id_{b},h)  \\
  \btensor(f,h)\lambda &  \rho  \\};
\path[->, font=\scriptsize]
(b-1-1) edge node[auto]{\(\scriptstyle \btensor((f,\id_b),h)\) } (b-1-2)
(b-1-1)	edge node[auto]{\(\scriptstyle \btensor((l,\lambda \circ l),h)\)} (b-2-1)
(b-2-1) edge node[auto]{\(\scriptstyle \btensor((r,\rho\circ r),h)\)} (b-2-2)
(b-1-2) edge node[auto]{\(\scriptstyle \btensor((g,\id_s),h)\)} (b-2-2) ;
\end{tikzpicture}. 
\end{equation*}  
The source of this cube is therefore cartesian, and it can be written as 
\begin{equation*}
\begin{tikzpicture}
\matrix (b) [matrix of math nodes, row sep=3em,
column sep=3em, text height=1.5ex, text depth=0.25ex]
{ P_{f,h} &  P_{\id_{b},h}  \\
  P_{\lambda,h} &  P_{\rho,h}  \\};
\path[->, font=\scriptsize]
(b-1-1) edge node[auto]{\(\scriptstyle \btensor((f,\id_b),h)^0\) } (b-1-2)
(b-1-1)	edge node[auto]{\(\scriptstyle \btensor((l,\lambda \circ l),h)^0\)} (b-2-1)
(b-2-1) edge node[auto]{\(\scriptstyle \btensor((r,\rho\circ r)^0,h)\)} (b-2-2)
(b-1-2) edge node[auto]{\(\scriptstyle \btensor((g,\id_s),h)^0\)} (b-2-2) ;
\end{tikzpicture}, 
\end{equation*}
which exhibits \(\btensor((g,\id_b),h)^0\) as a pushout of \(\btensor((f,\id_b),h)^0\).  However, we can easily see that the target of the square 
\begin{equation*}
\begin{tikzpicture}
\matrix (b) [matrix of math nodes, row sep=3em,
column sep=3em, text height=1.5ex, text depth=0.25ex]
{\btensor(f,h) &  \btensor(\id_{b},h) \\};
\path[->, font=\scriptsize]
(b-1-1) edge node[auto]{\(\scriptstyle (f,\id_b)\) } (b-1-2)
;
\end{tikzpicture}
\end{equation*}
is exactly the arrow
\begin{equation*}
\begin{tikzpicture}
\matrix (b) [matrix of math nodes, row sep=3em,
column sep=3em, text height=1.5ex, text depth=0.25ex]
{\Box(b,Y) &  \btensor(b,Y)  \\};
\path[->, font=\scriptsize]
(b-1-1) edge node[auto]{\(\scriptstyle \Box(\id_b,\id_b)\) } (b-1-2);
\end{tikzpicture}, 
\end{equation*}
which is the identity map \(\id_{\Box(b,Y)}\).  Moreover, by \eqref{idcubes}, we see that \(\btensor(\id_b,h)=\id_{\Box(b,Y)}\).  This means that \(\btensor((f,\id_b),h)^0=\btensor(f,h)\), and therefore that \(\btensor((g,\id_b),h)^0\) can be expressed as a pushout of such a map.
\end{obs} 
\begin{lemma}\cite{ourythesis}*{Lemma 3.10} Given classes of maps \(J_i\) of \(\C_i\), we have a containment \(\btensor(J_0,\dots,\operatorname{Cell}(J_i),\dots,J_n) \subseteq (\operatorname{Cell}(\btensor(J_0,\dots,J_n))\).
\end{lemma}
\begin{proof} Continuing as we have above, we prove the case where \(i=0\), since the other proofs are identical.  Let \(f_0:A_0\to B_0\) be a fixed element of \(\operatorname{Cell}(J_0)\).  Then we may give \(f_0\) as the induced map \(A_0\to \varinjlim(F)=B_0\)  over an ordinal-indexed diagram \(F:\alpha\to \C_0\) such that \(F(j\to j+1)\) is the pushout of a map in \(J_0\) for every \(j\in \alpha\) and such that the induced map \(\varinjlim(\eval[2]{F}_{j<k})\to F(i)\) is an isomorphism for every limit ordinal \(k\in \alpha\).   Using the fact that the colimit of this diagram is \(B_0\), the cone diagram induces a diagram \(F^\prime:\alpha\times [1] \to \C_0\), which is the natural transformation diagram:
\begin{equation*}
\begin{tikzpicture}
\matrix (b) [matrix of math nodes, row sep=3em,
column sep=3em, text height=1.5ex, text depth=0.25ex]
{ F(0) &  F(1) &\dots  \\
  \varinjlim(F) &  \varinjlim(F) & \dots  \\};
\path[->, font=\scriptsize]
(b-1-1) edge  (b-1-2)
(b-1-2) edge  (b-1-3)
(b-1-1)	edge  (b-2-1) 
(b-2-1) edge  (b-2-2)
(b-2-2) edge  (b-2-3)
(b-1-2) edge  (b-2-2) 
(b-1-3) edge  (b-2-3);
\end{tikzpicture}. 
\end{equation*}
Fixing \(g=(g_1,\dots,g_n)\) with \(g_i\in J_i\) for \(1\leq i\leq n\), we may form a functor \(\btensor(F,g)\).  The colimit of this functor is created by \(\btensor(\bullet,g)\) and is exactly \(\btensor(\id_{B_0},g)\), which is the identity by \eqref{idcubes}.  Since the bottom of this square is also the identity, we may identify \(\btensor(f_0,g)\) with the induced map \(P_{f_0,g}\to P_{\id_{B},g}\).  However, since \(\id_B=\varinjlim F'\), we have that \(\btensor(\varinjlim(F'),g) = \varinjlim \btensor(F',g)\), but this immediately implies that the induced map \(P_{f_0,g}\to  P_{\id_{B},g}\) is the transfinite composite of the maps \(\btensor(F^\prime(k<k+1),g)^0,\) which are easily seen to be the maps \(\phi_k^{k+1}:P_{F^\prime(k),g}\to P_{F^\prime{k+1},g}\).  

However, each of these maps arises from the process described in the previous observation, from a pair of commutative squares, the cocartesian one arising from the fact that \(F(k<k+1)\) is a pushout of a map \(h^{k+1}_k\) belonging to \(J_0\) and the second one arising from the square \(F'(k<k+1)\), which means that \(\phi_k^{k+1}\) is a pushout of \(\btensor(h^{k+1}_k,f\).  Then this proves that \(\btensor(f_0,g)\in (\operatorname{cell}(\btensor(J_0,\dots,J_n)\) as claimed.
\end{proof}

\subsection{Corner tensors in multicategories}
\begin{defn} Let \(\cat{Rex_c}\) denote the symmetric sub-multicategory of \(\cat{Cat}\) whose objects are the finitely cocomplete categories, and whose \(k\)-morphisms are those \(k\)-fold functors preserving finite connected colimits independently argument by argument.  

Let \(\cat{Rex}\) be the submulticategory of \(\cat{Rex_c}\) with the same objects but whose \(k\)-morphisms preserve \emph{all} finite colimits argument by argument. 
\end{defn}
%\begin{obs} Any category \(\C\) in \(\Rex\)
%\end{obs}

\subsection{Horns as corner tensors}