\section{Iterated wreath products and the category $\Theta$}
\label{sec:wreath}

We will make use of two equivalent definitions of the category \(\Theta\) of cell objects: The first definition, covered in this section is due to Berger in \cite{berger-iterated-wreath}, where he defines it to be the filtered union of wreath powers of the simplex category \(\Delta\) along the inclusion maps \(\Delta^{\wr n}\cong \Delta^{\wr n}\wr \ast \hookrightarrow \Delta^{\wr n+1}\).  This section will give a quick review of this theory.  Some of the exposition in this section is based on \cite{cisinski-decalage}, and the author makes no claims of originality in this section. 

\subsection{The category $\Gamma$}

Recall that Segal's category, \(\Gamma\), is defined as follows: The objects are the (possibly empty) sets \(\Gamma_m=\{x\in \mathbf{N}\colon 1\leq x\leq m\}\) for each \(m\in \mathbf{N}\), and morphisms \(\Gamma_m\to \Gamma_n\) are functions \(f\colon \Gamma_m\to 2^{\Gamma_m}\) such that \(f(a)\cap f(b) = \emptyset\) if \(a<b\)).  Given morphisms, \(f\colon \Gamma_m\to \Gamma_n\) and \(g\colon \Gamma_p\to \Gamma_m\), we define the composite \(f\circ g  \colon \Gamma_p\to \Gamma_n\) by letting \((f\circ g)(s)=\bigcup_{t\in f(s)}g(t)\).  It is left as an exercise to the reader to show that this law of composition is indeed associative.  

\begin{prop}The category \(\Gamma\) is equivalent to the category \(\overcat{0}{\cat{Fin}}^\op\), where \(\cat{Fin}\) is defined to be the category of finite sets, and where \(0\) denotes the set with a single element.
\end{prop}
\begin{proof}
It is immediate that \(\overcat{0}{\cat{Fin}}\) is equivalent to the subcategory of \(\cat{Set}\) whose objects are the sets \(n=\{x\in \mathbf{N}\colon 0\leq x\leq n\}\) and whose morphisms are those functions \(f\colon n\to m\) such that \(f(0)=0\).  

Given such a function, we define a map \(\Gamma_f:\Gamma_m\to \Gamma_n\) by the rule \(x\mapsto f^{-1}(x)\).  Conversely, given a map \(f\colon \Gamma_m\to \Gamma_n\), we define the subset \(f(0)\subseteq \Gamma_n\) to be the complement \[f(0)=\Gamma_n - \bigcup_{s\in \Gamma_m} f(s).\]  

Then we define the function \([f]:n\to m\) by the rule \([f](s)=j\) where \(j\) is the unique number \(0\leq j\leq m\) such that \(s\in f(j)\).  It is clear that\([\Gamma_f]=f\) and \(\Gamma_{[g]}=g\), so this determines an anti-equivalence of categories.
\end{proof}

\subsection{The categorical wreath product}

Let \(A\) be a category, and let \(F\colon B\to \Gamma\) be an object of \(\overcat{\cat{Cat}}{\Gamma}\).  

Then we define the \dfn{wreath product} \(B\wr A\) as follows: The objects of \(B\wr A\) are pairs \((b, \{a_i\}_{i\in F(b)})\) comprising an object \(b\) of \(B\) and a family of objects of \(A\) indexed by the elements of \(F(b)\).  

A morphism \((b, \{a_i\}_{i\in F(b)})\to (b', \{a'_i\}_{i\in F(b')})\) is given by the data of a pair \((f, \{\eta_{ij}\})\) comprising
\begin{enumerate}
\item [(i)] a morphism \(f \colon b\to b'\) of \(B\), and
\item [(ii)] a morphism \(\eta_{ij}:c_i\to d_j\) for each pair \(i,j\) such that \(i\in F(b)\) and \(j\in F(f)(i)\)
\end{enumerate}

It is left as an easy exercise to show that the composition of two such maps obtained in the obvious way is indeed associative.

If \(G\colon (B',F')\to (B,F)\) is a functor over \(\Gamma\), and \(\Psi\colon A'\to A\) is any functor, we obtain a functor \(G\wr F\colon B' \wr A'\to B\wr A\) defined on objects by the rule \[(b',(a'_i)_{i\in F'(b')})\mapsto (G(b),(\Psi(a'_i))_{i\in F(G(b'))}\] (which makes sense since \(F\circ G=F'\)) and is defined on morphisms in the obvious way. 

\begin{defn} We say that a category \(C\) is \dfn{semi-additive} if it admits finite products and a null object.  A morphism between such categories is a finite-limit preserving functor
\end{defn}

\begin{prop}The category \(\Gamma\wr A\) is the free semi-additive category on A, and there exists a canonical functor \(\alpha:\Gamma\wr\Gamma \to \Gamma\) sending an n-tuple of objects \(\Gamma_n(\Gamma_{i_1},\dots \Gamma_{i_n})\) to their sum, \(\Gamma_{i_1+\dots+i_n}\).  
\end{prop}
\begin{proof} 
Since \(\Gamma_0\) is a null object for \(\Gamma\), we see that \(\Gamma\wr A\) has null object equal to \(\Gamma_0()\).  We also have for any two objects \(X=\Gamma_i(a_1,\dots a_i)\) and \(Y=\Gamma_j(b_1,\dots b_j)\) an object \(\Gamma_{i+j}(a_1,\dots,a_i,b_1,\dots b_j)\), which, when equipped with the two projections \(\Gamma_{i+j}(a_1,\dots,a_i,b_1,\dots b_j)\to \Gamma_i(a_1,\dots,a_n\) and \(\Gamma_{i+j}(a_1,\dots,a_i,b_1,\dots b_j)\to \Gamma_j(b_1,\dots,b_j\) is easily checked to be a cartesian product of \(X\) and \(Y\).  It also admits a unique embedding \(A\hookrightarrow \Gamma\wr A\) defined on objects by the rule \(a\mapsto \Gamma_1(a)\), so every nonzero object of \(\Gamma\wr A\) is uniquely a product of objects of the form \(\Gamma_1(a)\). 

Since \(\Gamma\cong \Gamma\wr \ast\), and \(\Gamma\wr \ast\) is semi-additive, there exists a unique finitely continuous functor \(\Gamma\wr \Gamma\to\Gamma\wr \ast \cong \Gamma\), which sends an object 
\(\Gamma_n(\Gamma_{i_1},\dots \Gamma_{i_n}) = \prod_{j=1}^n\Gamma_1(\Gamma_{i_j})\) 
to the object \(\prod_{j=1}\Gamma_{i_j}=\Gamma_{\sum_{j=1}^n i_j}\).  
\end{proof}

\begin{prop} The wreath product is a monoidal product for the category \(\overcat{\cat{Cat}}{\Gamma}\) with monoidal unit given by the functor classifying \(\Gamma_1\), \(e_{\Gamma_1}:*\to \Gamma\).
\end{prop}
\begin{proof}
We see that \((*,e_{\Gamma_1})\wr (B,\lambda_B) \to \Gamma\wr\Gamma\) sends the objects \(\ast(b)\) to the objects \(\Gamma_1(\lambda_B(b))\), which maps under \(\alpha\) to the object \(\Gamma_{\lambda_B(b)}\) of \(\Gamma\).  Similarly 
\((B,\lambda_B)\wr (*,e_{\Gamma_1})\) sends the objects \(b(\ast,\ast,\dots,\ast)\) to the objects \(\lambda_B(b)(\ast,\ast,\dots,\ast),\) which map under \(\alpha\) to the objects \(\Gamma_{\sum_{i=1}^{\lambda_B(b)} 1}\), which are precisely the objects \(\Gamma_{\lambda_B(b)}.\)

Let \((A,f_A)\), \((B,f_B)\), \((C,f_C)\) be categories over \(\Gamma\) (we will suppress the functors \(f_X\) unless otherwise noted) to.  To prove the associativity of \(\wr\), we see that there is an isomorphism of categories, natural in \(A,B,C\), \(\alpha_{ABC}:(A\wr B)\wr C\to A \wr (B\wr C) \) where the object 
\((a, \{b_i\}_{i\in f_A(a)})(\{c_j: j\in F_b(f_A(a))\})\) is sent to the object 
\((a,\{b_i, \{c_j\}_{j\in f_B(b_i)}\}_{i\in f_A(a)}).\)

The definition on morphisms can be extracted from the definition on objects by reindexing, and we leave an explicit description of this reindexing as an exercise.  Naturality in \(A,B,C\) follows from the functoriality of the wreath product. From this, we obtain a commutative square in \(\cat{Cat}\) 

\begin{equation*}
\begin{tikzpicture}
\matrix (b) [matrix of math nodes, row sep=3em,
column sep=3em, text height=1.5ex, text depth=0.25ex]
{ (A\wr B)\wr C & A\wr(B\wr C) \\
   (\Gamma\wr \Gamma)\wr \Gamma &  \Gamma\wr(\Gamma\wr\Gamma) \\};
\path[->, font=\scriptsize]
(b-1-1) edge (b-1-2)
        edge (b-2-1)
(b-2-1) edge (b-2-2)
(b-1-2) edge (b-2-2);
\end{tikzpicture},
\end{equation*}
from which it follows that \(\alpha_{ABC}\) is indeed a morphism over \(\Gamma\) for all triples \(A,B,C\) if and only if the isomorphism \(\alpha_{\Gamma\Gamma\Gamma}\) is a morphism over \(\Gamma\).  However, it is easy to see that this holds, ultimately, by the generalized associativity of iterated addition in \(\mathbf{N}\).  
\end{proof}

\subsection{Infinite wreath products}\label{infwreath}

Iterating the wreath product construction on a category \(F_B:B\to \Gamma\) over \(\Gamma\), we obtain by recursion a definition of the \(n^\mathrm{th}\) wreath power \(B^{\wr n+1}=B \wr B^{\wr n}\).  Suppose further that \(B\) is equipped with a functor \(e_b\colon \ast \to B\) classifying an object \(b\) of \(B\). Then by iterating the wreath product construction on \(e_b:\ast\to B\), we construct the following data by recursion:
\begin{enumerate}
\item[(i)] Let \(B_0= \ast\), and let \(\iota_0=e_b\).
\item[(ii)] Let \(B_{i+1}=B\wr B_i\), and let \(\iota_{i+1}=\id_{B}\wr \iota_{i}:B\wr B_{i}\to B\wr B_{i+1}\).
\end{enumerate}
Since \(B\wr \ast\) is canonically isomorphic to \(B\), we obtain a diagram, \(T_{B,b,F_B}:\mathbf{N}\to \cat{Cat}\):
\begin{equation*}
\begin{tikzpicture}
\matrix (a) [matrix of math nodes, row sep=3em,
column sep=3em, text height=1.5ex, text depth=0.25ex]
{ B_0 & B_1 & B_2 & \dots & B_n & B_{n+1} & \dots \\};
\path[right hook->, font=\scriptsize]
(a-1-1) edge node[auto]{\(\scriptstyle \iota_0\)} (a-1-2)
(a-1-2) edge node[auto]{\(\scriptstyle \iota_1\)} (a-1-3)
(a-1-3) edge node[auto]{\(\scriptstyle \iota_2\)} (a-1-4)
(a-1-4) edge node[auto]{\(\scriptstyle \iota_{n-1}\)} (a-1-5)
(a-1-5) edge node[auto]{\(\scriptstyle \iota_n\)} (a-1-6)
(a-1-6) edge node[auto]{\(\scriptstyle \iota_{n+1}\)} (a-1-7);
\end{tikzpicture}
\end{equation*}
 
We then define \(C(B,b,F_B)=C(B,b)=\varinjlim T_{B,b,F_B}\) as the colimit of this system.  

\subsection{The simplex category $\Delta$}

Recall that the simplex category, \(\Delta\), is defined to be the skeleton of the full subcategory of \(\cat{Cat}\) spanned by the finite nonempty
linearly-ordered sets (regarded as categories).  The objects of \(\Delta\) are isomorphism classes of linearly ordered sets, where \([n]\) denotes the
class of the linearly-ordered set \(\{0<\dots<n\}\).  In fact, we may identify the skeleton with the full subcategory spanned by such sets.  In the sequel,
we will make this identification without timidity, justified by the fact that there is at most one isomorphism between any two linearly-ordered sets.

Following Rezk in \cite{rezk-theta-n-spaces}, we call a map \(f \colon [n]\to [m]\) in \(\Delta\) \dfn{sequential} if 
\[
f(i-1)+1 \geq f(i),\qquad \text{\(1\leq i \leq n\)}.
\]

We say that an object \(\gamma:[1]\to [n]\) of \(\overcat{[1]}{\Delta}\) is an interval if \(\gamma(0)=0\) and \(\gamma(1)=n\), and we say that an interval is strict if the map \(\gamma\) is also injective.  Let \(\mathcal{D}^1\) denote the full subcategory of \(\overcat{[1]}{\Delta}\) spanned by the strict intervals.  We denote a strict interval whose underlying simplex is \([n]\) by \(|n|\), and we denote the image of the inclusion \(\gamma\) by \(\partial |n|\).  

We have a functor \(q:\mathcal{D}^1\to \overcat{0}{\cat{Fin}}\) defined on objects by the formula \[|n+1|\mapsto |n+1|/\partial|n+1|=n\] and defined on morphisms by the universal property of quotients.  This gives us a functor \(q^\op:  (\mathcal{D}^1)^\op \to \overcat{0}{\cat{Fin}}^\op\cong \Gamma\).  However, the category \(\mathcal{D}^1\) is isomorphic to \(\Delta^{op}\) by the functor \([n]\mapsto \Delta([n],[1])=|n+1|\), the confirmation of which we leave to the reader.  

\label{segfun}We write \(F_\Delta\) for the induced functor \(\Delta\to \Gamma\).  We see that clearly, \(F_\Delta([m])=\Gamma_m\), and given \(f:[n]\to [m]\), we compute \(F_\Delta(f)(i)\) for \(i\in \Gamma_n\).  First, we obtain a morphism \[f^\ast:|m+1|=\Delta([m],[1])\to \Delta([n],[1])=|n+1|,\] which descends to a morphism \(q(f^\ast):n=q(|n+1|)\to q(|m+1|)=m\), so \(F_\Delta(f)(i)=\Gamma_{q(f^\ast)}(i)=(q(f^\ast))^{-1}(i)\), but \(q(f^{\ast})(j) = i\) if and only if \(f^\ast(j)=i\).  Let \(c_j:[m]\to [1]\) be the unique morphism such that \(j=\inf(c_j^{-1}(\{1\})\).  Then \(f^\ast(j)=i\) holds if and only if \(i=\inf((c_j\circ f)^{-1}(1))\) if and only if \(f(i-1)<j\leq f(i)\).  Then \(F_\Delta(f)(i)=\{j: f(i-1)<j\leq f(i)\}\).  This gives us an explicit description of the functor \(F_\Delta:\Delta\to \Gamma\).  Combining this with the definition of the wreath product, we obtain:

\subsection{The category $\Delta \wr \C$}

Let \(\C\) be a category.  Then applying the wreath product construction with respect to the functor \(p\colon\Delta\to \Gamma\), we may describe the category \(\Delta \wr \C\) explicitly as follows: An object of \(\Delta \wr \C\) is given by the data of a pair \((n,(c_1,\dots c_n))\), written \([n](c_1,\dots,c_n\), 
where \(n\in \mathbf{N}\) and \((c_1,\dots,c_n) \in Ob(\C^{\times n})\).  

A morphism \([n](c_1,\dots,c_n)\to
[m](d_1,\dots,d_m)\) is given by the data of a pair \((f, \{\eta_{ij}\})\) comprising
\begin{enumerate}
\item [(i)] a morphism \(f \colon [n]\ra [m]\) of \(\Delta\), and
\item [(ii)] a morphism \(\eta_{ij}:c_i\to d_j\) for each pair \(i,j\) such that \(f(i-1)<j\leq f(i)\)
\end{enumerate}

In general, for any category \(\C\), we will call the category \(\Delta \wr \C\) the \dfn{\(\Delta\)-suspension} of \(\C\).  
We will define \(\Theta\) to be \(C(\Delta,[0])\). 

\section{Strong generators and completions}

We will find it extremely useful to sharpen Cisinski's theory \cite{cisinski-book}*{1.4}with respect to how localizers are generated with respect to simplicial completions and how to deal with regularity \cite{cisinski-book}*{3.4} with these generators.

\subsection{Simplicial generators for localizers}

Let \(\C\) be a small category.  We let \(\mathsf{W}_\infty\) denote the \(\C\times \Delta\)-localizer generated by the maps \(X\times \Delta_n\to X\times \Delta_0\) for every presheaf \(X\) on \(\C\) and every \(n\geq 0\).  
\begin{prop}[\cite{cisinski-book}*{Corollary 2.3.7}] The \(\C\times \Delta\)-localizer \(\mathsf{W}_\infty\) is accessible.
\end{prop}   
\begin{proof}See the proof in \cite{cisinski-book}.
\end{proof}
\begin{defn}
We say that a class of maps \(S\) in \(\psh{\C}\) is a weak class of irregular generators for a \(\C\)-localizer \(\W\) if \(\W(S)=\W\).

We say that a class of maps \(S\) in \(\psh{\C\times \Delta}\) is a class of \dfn{simplicial irregular generators} for a localizer \(\W\) if the \(\C\times \Delta\)-localizer \[\mathsf{W}(S\times \Delta_0 \cup \mathsf{W}_\infty)\] is exactly the simplicial completion of \(\W\).  

We say that a class of maps \(S\)  in \(\psh{\C}\) is a class of \dfn{strong irregular generators} for \(\W\)  if the class \(S\times \Delta_0\) of maps of the form \(f\times \Delta_0\) where \(f\in S\) is a class of simplicial irregular generators for a \(\C\)-localizer \(\W\).
\end{defn}
\begin{prop}If \(S\) is a class of strong irregular generators for a \(\C\)-localizer \(\mathsf{W}\), then \(\W=\W(S)\).
\end{prop}
\begin{proof}This follows immediately from \cite{cisinski-book}*{Proposition 2.3.30}.  
\end{proof}
There is a useful and na\"ive way to strengthen classes of weak irregular generators to classes of strong irregular generators:
\begin{prop} If \(S\) is a class of weak irregular generators for a localizer \(\W\), then \(S\cup \operatorname{cart}(\{\ell:L\to e\})\) is a strong class of generators, where \(L\) is the subobject classifier of \(\C\), and \(\operatorname{cart}(\{\ell\})\) is the class of all maps \(X\times \ell:X\times L \to X\) where \(X\) is a presheaf on \(\C\).  
\end{prop}
\begin{proof}This again follows immediately from \cite{cisinski-book}*{Proposition 2.3.30}.
\end{proof}
\subsection{Strong regular generators for regular localizers}
To utilize this notion of strong generation for a localizer in the context of regular localizers, the following important proposition will be extremely useful:
\begin{prop}If \(S\) is a class of strong simplicial irregular generators for a \(\C\)-localizer \(\mathsf{W}\), then the simplicial completion of the regular completion \(\mathsf{R}(\W)\) of \(\W\) is the \(\C\times \Delta\)-localizer \(\W(S \cup \operatorname{R}(\W_\infty))\), where \(\mathsf{R}(\W_\infty)\) is the regular completion of \(\W_\infty\), which is precisely the class of objectwise weak homotopy equivalences of simplicial presheaves.
\end{prop}
\begin{proof}This follows easily from \cite{cisinski-book}*{Corollary 3.4.47}.
\end{proof}
Based on this proposition, we can give a slightly weaker notion of strong generation:
\begin{defn} 
We say that a class of maps \(S\) in \(\psh{\C}\) is a class of \dfn{weak regular generators} for a regular localizer \(\W\) if \(S\) is a class of weak irregular generators for some \(\C\)-localizer \(\W^\prime\) whose regular completion \(\mathsf{R}(\W^\prime)\) is exactly \(\W\). 

We say that a class of maps \(S\) in \(\psh{\C\times \Delta}\) is a class of \dfn{simplicial regular generators} for a regular localizer \(\W\) if \(S\) is a class of simplicial irregular generators for some \(\C\)-localizer \(\W^\prime\) whose regular completion \(\mathsf{R}(\W^\prime)\) is exactly \(\W\). 

We say that a class of maps \(S\)  in \(\psh{\C}\) is a class of \dfn{strong regular generators} for \(\W\)  if the class \(S\times \Delta_0\) of maps of the form \(f\times \Delta_0\) where \(f\in S\) is a class of simplicial regular generators for a \(\C\)-localizer \(\W\).

Unless otherwise noted, when \(\W\) is a regular localizer, a class of \dfn{strong generators} for \(W\) will always mean a class of strong \emph{regular} generators.   
\end{defn}
Then we easily obtain the following useful corollary:
\begin{cor} If \(S\) is a small set of strong generators for a regular localizer \(\W\), then the simplicial completion of \(\W\) is the class of weak equivalences of the left Bousfield localization of \(\psh{\C\times \Delta}_{\operatorname{inj}}\) at \(S\).   
\end{cor}
\begin{prop}\label{boostgenerators} If \(S\) is a class of weak regular generators for a localizer \(\W\), then \(S\cup \operatorname{cart}(\{\ell:L\to e\})\) is a strong class of regular generators, where \(L\) is the subobject classifier of \(\C\), and \(\operatorname{cart}(\{\ell\})\) is the class of all maps \(X\times \ell:X\times L \to X\) where \(X\) is a presheaf on \(\C\).  
\end{prop}
\begin{proof}This again follows immediately from from \cite{cisinski-book}*{Corollary 3.4.47}.
\end{proof}


\section{A model structure on $\widehat{\Delta \wr \C}$}\label{sec:cishelp}
We thank Denis-Charles Cisinski for his invaluable help with the formulation of this section. We will give a model structure whose fibrant objects are models for categories weakly enriched in the homotopy theory of \(\W\)-fibrant presheaves of sets on \(\C\) whenever \((\C,\W)\) a pair comprising a small category \(\C\) together with a fixed accessible cartesian regular \(\C\)-localizer, \(\W\).  For now, we fix the small category \(\C\).

\subsection{The intertwining functor $V_{\C}$}
For any category \(A\), we let \(Y_A:A\hookrightarrow \psh{A}\) denote the Yoneda embedding. Then we have an apparent pair of functors \[Y_{\Delta \wr \C}:\Delta \wr \C \hookrightarrow \psh{\C},\] the Yoneda embedding of \(\Delta \wr \C\), and by the functoriality of the wreath product, the \(\Delta\)-suspended Yoneda embedding of \(\C\), \[L=\id_\Delta\wr Y_\C:\Delta \wr \C \hookrightarrow \Delta \wr \psh{\C}\].  

We define the \dfn{\(\C\)-intertwiner} \(V_\C:\Delta \wr \psh{\C} \to \psh{\Delta \wr \C}\) to be the left Kan extension \(L_!(Y_{\Delta \wr \C})\) of \(Y_\C\) along \(L\).  Unless there is a risk of confusion, we will typically suppress the subscript \(\C\).

\subsection{Mapping objects}

For any \(\psh{\C}\)-enriched simplicial set \(X\), equipped with a pair of vertices \((x_0,x_1)\) of \(X,\) we will construct a mapping object \(X(x_0,x_1)\) of \(\psh{\C}\).

The following lemma is due to Rezk in \cite{rezk-theta-n-spaces}:
\begin{lemma}\label{leftadjointness}
Given any two families \(A_1,\dots, A_m\) and \(B_1,\dots, B_n\) of presheaves on \(\C\), the functor \(P:\psh{\C}\to \psh{\Delta\wr \C}\) defined by the formula \[X\mapsto V[n+1+m](A_1,\dots,A_m,X,B_1,\dots,B_n)\] is a parametric left adjoint, that is to say, the natural factorization \[P_0:\psh{\C}\to \overcat{P(\emptyset)}{\widehat{\Delta\wr \C}}\] of \(P\) through the forgetful functor \[U_0: \overcat{P(\emptyset)}{\psh{\Delta\wr \C}}\to \psh{\Delta\wr \C}\] admits a right adjoint. Further, we have that \[P(\emptyset)=V[m](A_1,\dots,A_m)\coprod V[n](B_1,\dots,B_m)\].   
\end{lemma}
\begin{proof}
Since we are taking the left Kan extension of the Yoneda embedding along \(L=\id_\Delta\wr Y_\C\), if we let  \(h_Z\), for any object \(Z\) of \(\Delta \wr \psh{\C}\), be the functor \(A\mapsto \Hom_{\Delta\wr\C}(A,Z)\) representing \(Z\),  we  obtain a simple formula for \(VZ\) as \(L^\ast(h_X)\) because the conical formula for the pointwise left Kan extension degenerates on the Yoneda embedding.

To see why this is true, notice that in the conical formula for the left Kan extension, we have that \[V(Z)=\varinjlim(\overcat{L}{Z}\to \Delta\wr\C\to \psh{\Delta\wr\C}),\] where \(\overcat{L}{Z}\) is the pullback \(\Delta\wr \C \to \Delta\wr\psh{\C} \leftarrow \overcat{\Delta\wr\psh{\C}}{Z}\).  However, by inspection, the category \(\overcat{L}{Z}\) is precisely the category of elements of the \(\Delta\wr\C\)-presheaf \(L^\ast (h_Z)\), so composing this diagram with the Yoneda embedding and taking a colimit is precisely the colimit of the category of elements of the presheaf \(L^\ast(h_Z)\), which just so happens to be \(L^\ast (h_Z)\) by Yoneda's lemma.  

Let \(a=[q](c_1,\dots,c_q\) be an object of \(\Delta\wr\C\).  Following Rezk in \cite{rezk-theta-n-spaces}, we see that the set of maps \(a\to L(X)\) belongs, can be divided into partitions corresponding to the partitions of \(\Hom_\Delta([q],[m+1+n])\), parameterized by the elements \(p\in \Hom_\Delta([q],[1])=\set{p}{0\leq p\leq q+1}\) as follows: 
\[G(p)=
\begin{cases}
\set{\delta}{\delta(0)\geq m+1}\qquad \text{if \(p=0\)}\\
\set{\delta}{\delta(p-1)\leq m, \delta(p)\geq m+1} \qquad \text{if \(1\leq p\leq q\)}\\
\set{\delta}{\delta(q+1)\leq m}\qquad \text{if \(p=q+1\)}
\end{cases},
\]
which decomposes the set \(\Hom_{\cellset}(a,L(X))\) into the factors \((S_0,\dots,S_{q+1})\), where the factor \(S_0\) is 
\[
\coprod_{\delta\in G(0)}\,
\prod_{i=1}^q\,\prod_{j=\delta(i-1)+1}^{\delta(i)} B_{j-(m+1)}(c_i) \approx
V[n](B_1,\dots,B_n)(\theta),
\]
the factor \(S_{q+1}\) is 
\[
\coprod_{\delta\in G(q+1)}\,
\prod_{i=1}^q\, \prod_{j=\delta(i-1)+1}^{\delta(i)} A_j(c_i) \approx
  V[m](A_1,\dots, A_m)(\theta),
\]
and the factor \(S_p\) for \(1\leq p \leq q\) is 
\[
\coprod_{\delta\in G(p)}
\left(\prod_{i=1}^{p} \,\prod_{j=\delta(i-1)+1}^{\min(\delta(i),m)} A_j(c_i)
\right) \times X(c_p) \times \left(\prod_{i=p}^q \,
  \prod_{j=\max(\delta(i-1),m)+2}^{\delta(i)} B_{j-(m+1)}(c_i)\right).
\]

It follows by inspection that the functor \(P_0\) preserves colimits and that \[P(\emptyset)=V[m](A_1,\dots,A_m)\coprod V[n](B_1,\dots,B_n).\]
\end{proof}
Since \(V[1](\emptyset)=\ast\coprod \ast\), the preceding lemma in the case where \(m=n=0\) gives us our desired right adjoint \(R:\overcat{V[1](\emptyset)}{\psh{\Delta\wr\C}}\to \psh{\C}\).  Given a \(\psh{C}\)-enriched simplicial set \(X\) together with a pair of vertices \((x_0,x_1)\) of \(X,\) we can take these data together to give a map \((x_0,x_1):V[1](\emptyset)\to X\), which give an object \(X,(x_0,x_1)\) of \(\overcat{V[1](\emptyset)}{\psh{\Delta\wr\C}}\).  Then we define \(X(x_0,x_1)=R(X,(x_0,x_1))\).  By functoriality, for any map \(f:X\to Y\) in \(\psh{\Delta\wr \C}\) and any pair of vertices \(x_0,x_1\), we obtain a natural map \(f_{x_0,x_1}:X(x_0,x_1)\to Y(f(x_0),f(x_1))\).  Indeed, it is for this reason that we call \(\psh{\Delta\wr\C}\) the category of \(\psh{\C}\)-enriched simplicial sets.
\subsection{Simplicial mapping objects and $A$-simplices}


\begin{defn} If \(S\) is a simplicial set equipped with a pair of vertices \((s_0,s_1):\Delta_0\coprod \Delta_0 \to X,\) we define \(S(s_0,s_1)\) to be the pullback of the diagram \[\Delta_0 \overset{(s_0,s_1)}{\to} S^{\partial \Delta_1} \leftarrow S^{\Delta_1},\] and we call it the \dfn{simplicial set of edges from \(s_0\) to \(s_1\)}.  This association is functorial in the category of bipointed simplicial sets and admits a right adjoint \(\Sigma\), the unreduced suspension functor, \[K\mapsto \left(\partial \Delta_1\to  K\times \Delta_1 \coprod_{K\times \partial \Delta_1} \Delta_0 \times \partial \Delta_1\right)\]
\end{defn}

\begin{lemma}For any \(\psh{\C}\)-enriched simplicial set \(X\) equipped with two vertices \((x_0,x_1)\), we may construct a simplicial presheaf \(\Map_X(x_0,x_1)\) on \(\C\), functorial in bipointed objects of \(\psh{\Delta\wr\C}\), such that \[X(x_0,x_1)=\Map_X(x_0,x_1)_0\] and \[\Hom(A,\Map_X(x_0,x_1))=\mathfrak{M}(A,X)(x_0,x_1).\]  Moreover, this functor arises from a cosimplicial enlargement of the functor \(A\mapsto \Delta_1[A]\).
\end{lemma}
\begin{proof}
We will show that \(\Sigma K [-]:\psh{\C}\to \psh{\Delta\wr \C}\) is a parametric left adjoint for any simplicial set \(K\).  It suffices to prove this when \(K\) is a simplex or empty, since \(\Sigma\) is well-known to be a parametric left adjoint.  The case when \(K\) is empty is clear, since \(\Sigma\emptyset=\Delta_0 \coprod \Delta_0\), and \((\Delta_0\coprod \Delta_0)[A]\) is just a coproduct of two vertices for every presheaf \(A\) on \(\C\).  

The case when \(K=\Delta_0\) is simply the functor \(\Delta_1[-]\), which is a parametric left adjoint by \eqref{leftadjointness}.  For \(K=\Delta_n\), we can decompose \(\Sigma K\) using the prism decomposition for the product \(\Delta_n\times \Delta_1\).  The prism decomposition presents \(\Delta_n\times \Delta_1\) as the colimit 
\[\varinjlim\left (\Delta_{n+1} \overset{\delta_n}{\leftarrow} \Delta_n \overset{\delta_n}{\rightarrow} \Delta_{n+1} \overset{\delta_{n-1}}{\leftarrow} \dots \overset{\delta_1}{\rightarrow} \Delta_{n+1}\right ). \]

When we take the pushout of the diagram \[\Delta_n\times \Delta_1 \leftarrow \Delta_n \times \partial \Delta_1\to \Delta_0 \times \partial \Delta_1,\] together with the prism decomposition, we find that \(\Sigma(\Delta_n)\) can be identified with the colimit of the diagram 
\[E_{n+1}^{n} \leftarrow E_{n}^{n-1} \rightarrow E_{n+1}^{n-1} \leftarrow E_n^{n-2} \to \dots \leftarrow E_n^1 \to E_{n+1}^1,\]
where \(E_n^i\) is the colimit of the diagram \[\Delta_0 \coprod \Delta_0 \leftarrow \Delta_{i-1} \coprod \Delta_{n-i} \hookrightarrow \Delta_n, \] where the map \(\Delta_{i-1}\hookrightarrow \Delta_n\) is the face spanned by the vertices \([0,\dots, i-1]\), and \(\Delta_{n-i}\hookrightarrow \Delta_n\) is the face spanned by the vertices \( [i,\dots,n]\).  Since \((-)[A]\) preserves colimits, it will suffice to show that \(E_n^i[-]\) is a parametric left adjoint for any pair \((n,i)\) such that \(1\leq i \leq n-1\).  

By the cocontinuity of \((-)[A]\), we may decompose \(E_n^i[-]\) as the pushout of the diagram \[\Delta_0\coprod \Delta_0 \leftarrow \Delta_{i-1} \coprod \Delta_{n-i}[-] \rightarrow \Delta_n[-].\]  However, it is clear from this construction that we may replace \(\Delta_{i-1} \coprod \Delta_{n-i}[-]\) by \[V[(i-1)+1+(n-i)](\ast,\dots,\ast,\emptyset,\ast,\dots,\ast)\] and \(\Delta_n[-]\) by \[V[(i-1)+1+(n-i)](\ast,\dots,\ast,-,\ast,\dots,\ast),\] since these are the parts of the functor that are killed in the pushout.  However, by \eqref{leftadjointness}, these functors are parametric left adjoints whose values on \(\emptyset\) are all exactly \(\partial \Delta_1=\Delta_1[\emptyset]\).  

It follows from this that we may define the aforementioned functor \[\Map:\overcat{\Delta_1[\emptyset]}{\psh{\Delta\wr \C}}\to \psh{\Delta\times \C}\] by the formula \[\Map_X(x_0,x_1)_n(c)=\Hom_{\overcat{\partial \Delta_1}{\psh{\Delta\wr \C}}}\left(\Sigma(\Delta_n)[c]_0^1, X_{x_0}^{x_1}\right),\] which is well-behaved since the functor \(\Sigma(K)[-]\) is a parametric left adjoint for every simplicial set \(K\).  We can see that \(\Hom(A,\Map_X(x_0,x_1))=\mathfrak{M}(A,X)(x_0,x_1)\) by unraveling the definitions.
\end{proof}

\subsection{The $\Delta\wr \C$-localizer $\mathsf{W}_{\Sc}$}
We begin with a small warning regarding notation:

\begin{note} Given a family of objects \(c=(c_1,\dots,c_n)\) of \(\C\), we will denote the presheaf \(Y_{\Delta\wr \C}([n](c_1,\dots, c_n))\) by \(\Delta_n[c]\).  Similarly, for a family of presheaves \((A_1,\dots,A_n)\) on \(\C\), we will denote \(V[n](A_1,dots,A_n)\) simply by \(\Delta_n[A]\).  We warn the reader that when \(A\) is simply a presheaf on \(\C\), this notation is used to mean \(V[n](A,\dots,A),\) but we are quite confident that the reader will be able to sort out which means which from context.  We just thought we'd let the reader know as a matter of courtesy.
\end{note}

\begin{defn}
Given a family \(c=(c_1,\dots,c_n)\) of objects of \(\C\), we define \dfn{Segal core} of the \(c\)-simplex \(\Delta_n[c]\) to be the \(\psh{\C}\)-enriched simplicial set 
\[\Sc_n[c]=\varinjlim\left(\Delta_1[c_1] \overset{\delta_0}{\leftarrow} \Delta_0 \overset{\delta_1}{\to} \dots \overset{\delta_0}{\leftarrow} \Delta_0 \overset{\delta_1}{\to} \Delta_1[c_n]\right ).\]
\end{defn}
\begin{defn} We define \(\mathsf{W}_{\Sc}\) to be the regular completion of the \(\Delta\wr\C\)-localizer generated by the class comprising the Segal core inclusions \(\Sc_n[c]\hookrightarrow \Delta_n[c]\) for any family of objects \(c=(c_1,\dots,c_n)\) of \(\C\). 
\end{defn}
\begin{lemma} The \(\psh{\C}\)-enriched simplicial set \(J=J[e]\), where \(J\) is the simplicial set obtained by taking the nerve of the strictly contractible groupoid \(G_2\) with two objects, is an injective object in \(\psh{\Delta\wr\C}\). 
\end{lemma}
\begin{proof}  The functor \(p:\Delta\wr \C \to \Delta=\Delta\wr \ast,\) induced by the terminal functor \(\C\to \ast,\) gives rise to an adjunction \[p_!:\psh{\Delta\wr\C}\leftrightarrows\psh{\Delta}:p^*.\]  We can see easily that \(p^*(X)=X[e]\) for any simplicial set \(X,\) because the functor \(p^*\) itself admits a right adjoint, which is \(\mathfrak{M}(e,-)\).  Then \(J[e]=p^*J=p^*\mathfrak{N}_\Delta(G_2),\) so it will suffice to show that \(p^*\mathfrak{N}_\Delta\) sends trivial fibrations in the natural model structure on \(\cat{Cat}\) to trivial fibrations of \(\psh{\C}\)-enriched simplicial sets.

However, this is equivalent to asking that the left adjoint of this functor sends monomorphisms of \(\psh{\C}\)-enriched simplicial sets to cofibrations between categories.  However, cofibrations in \(\cat{Cat}\) are just functors that induce injections on sets of objects.  We leave the easy proof of this fact to the reader.  
\end{proof}
\begin{cor}For every \(\psh{\C}\)-enriched simplicial set \(X\), the canonical map \(J\times X\to X\) is a trivial fibration, and in particular, belongs to \(\mathsf{W}_{\Sc}\).
\end{cor}
\begin{proof} Since \(J\) is an injective object, the map \(J\to e\) is a trivial fibration, which means that the map \(X\times J\to X\) is a trivial fibration as well, and therefore, it follows that \(X\times J\to X\) belongs to \(\mathsf{W}_\wr,\) since localizers contain all trivial fibrations.  
\end{proof}
\begin{cor}\label{ssetadjunction}The functor \((-)[e]:\psh{\Delta}\to \psh{\Delta\wr \C}\) is a left Quillen functor when \(\psh{\Delta}\) is equipped with the Joyal model structure and when \(\psh{\Delta\wr\C}\) is equipped with the Cisinski model structure generated by \(\mathsf{W}_{\Sc}\).  
\end{cor}
\begin{proof} Since the functor \((-)[e]=p^*\) admits an exceptional left adjoint, it necessarily preserves monomorphisms.  For this functor to preserve weak equivalences, we may equivalently show that the preimage of \(\mathsf{W}_\wr\) contains the Joyal weak equivalences. We can show that this is the case, then, by showing that the preimage is itself a \(\Delta\)-localizer containing the spine inclusions, which are known to generate the Joyal weak equivalences.  However, by \cite{cisinski-book}*{Proposition 1.4.20}, the preimage forms a \(\Delta\)-localizer provided that there exists some functorial cylinder \(\mathfrak{I}=(I,\partial^0,\partial^1,\sigma)\) of \(\Delta\) such that \(\sigma_X[e]:(I\otimes X)[e]\to X[e]\) is belongs to \(\mathsf{W}_\wr\) for every simplicial set \(X\).  

Since the functor \((-)[e]\) preserves products, again, since it admits a left adjoint, we see that the projection \((X\times J)[e]\to X[e]\) is exactly \(X[e]\times J[e]\to X[e]\), which belongs to \(W_\wr\) by the previous corollary.  This implies that the preimage of \(W_\wr\) indeed forms a \(\Delta\)-localizer, and this localizer clearly contains the spine inclusions, since these are mapped to Segal cores.
\end{proof}
We need to make use of a technical but straightforward lemma in order to obtain the upshot:
\begin{lemma}\label{scompcart} If \(\W\) is an accessible regular \(\C\)-localizer for a small category \(\C\), then \(\W\) is cartesian if and only if its simplicial completion\(\W_\Delta\) is cartesian.  
\end{lemma}
\begin{proof} Suppose \(\W\) is cartesian.  Then since the localizer is regular, its simplicial completion \(\W_\Delta\) is the class of weak equivalences obtained as the weak equivalences of the left Bousfield localization of the injective model structure on simplicial presheaves at the set \(S\times \Delta_0= \{s\times \Delta_0: s\in S\}\) for some set of maps \(S\) generating \(\W\).  Since the class of weak equivalences of the injective model structure is cartesian, it suffices to show that for any simplicial presheaf \(T\) on \(\C\) and any map \(s:A\to B\) in \(S\), the map \(s\times T: A\times T \to B\times T\) belongs to \(\W_\Delta\).  By regularity, \(T\) is the homotopy colimit of its category of elements, so \(s\times T\) is the homotopy colimit of maps of the form \(s\times (c \times \Delta_i)\), where the \(c\times \Delta_i\to T\) is a section of \(T\) for some object \((c,i)\) of \(\C\times \Delta\).  Then we have that \[s\times (c \times \Delta_i)
= (s\times c)\times \Delta_i,\] and since \(s\times c\) belongs to \(\W\) for every \(c\) in \(\C\), all of these maps are objectwise \(\W\)-equivalences and therefore belong to \(\W_\Delta\).  Since \(s\times T\) is the homotopy colimit of a diagram of weak equivalences, it is itself a weak equivalence and therefore, \(\W_\Delta\) is cartesian.  

The converse follows immediately from \cite{cisinski-book}*{Proposition 2.3.37}.  \end{proof}
\begin{lemma}\label{weakcats} The regular \(\Delta\wr \C\)-localizer \(\mathsf{W}_{\Sc}\) generated by the Segal cores is accessible and cartesian.  Moreover, it is strongly generated by the set of maps comprising the Segal core inclusions and the map \(j:J\to e\).  
\end{lemma}
\begin{proof} 
The first assertion is proven in two separate parts, since by \eqref{boostgenerators}, we note that \(\W_{\Sc}\) is strongly generated by the class \(\Sc \cup \operatorname{cart}(\{j\})\), where \(\Sc\) denotes the set of Segal cores.  Then we first show that \(\Sc\times \Delta_0\) generates a cartesian \(\Delta \wr \C \times \Delta\)-localizer.     
 
This is exactly the content of \cite{rezk-theta-n-spaces}*{Theorem 6.6} because we are looking at the regular completion, which means that we are Bousfield localizing the class of discrete Segal cores over the injective model structure. 

However, the reader should beware that the proof depends on \cite{rezk-theta-n-spaces}*{Proposition 6.4}, which was left uncorrected in the most recent revision of the paper.  The proof stated there is based on an incorrect statement from the published revision, and the author had forgotten to update it in the correction.  However, the proof of \eqref{coversweak} later in this paper can easily be modified to give a correct proof of that assertion. 

That the cartesian property holds for the whole simplicial completion is a corollary of \eqref{scompcart}, since this implies that \(\operatorname{cart}(\{j\})\times \Delta_0\) generates a cartesian \(\Delta \wr \C \times \Delta\)-localizer, and by \cite{cisinski-book}*{Corollary 1.4.19b}, \[\W(\Sc\times \Delta_0 \cup \mathsf{R}(\W_\infty))\cup \W(\operatorname{cart}(\{j\})\times \Delta_0 \cup \mathsf{R}(\W_\infty))\] generates a cartesian \(\Delta \wr \C \times \Delta\)-localizer, since each of the two parts generate cartesian \(\Delta \wr \C \times \Delta\)-localizers.  This implies by \eqref{scompcart} that \(\W_{\Sc}\) is indeed cartesian.

The second claim the real content of \cite{rezk-theta-n-spaces}*{Proposition 7.21}, and we refer the reader to the proof given there.
\end{proof}


\begin{lemma}The functor \(\Sigma(-)[A]:\Delta \to \psh{\Delta\wr\C}\) is a functorial cosimplicial resolution for the functor \(\Delta_1[A]\) \(\C\) associated with the localizer \(\mathsf{W}_{\Sc}\).  Moreover, this same cosimplicial resolution is also a cosimplicial resolution for \((\Delta_1)[A]\) viewed as an object in the coslice category under \(\Delta_1[\emptyset]\).
\end{lemma}
\begin{proof} First, we can see that the map \(\Delta[1]\to \Sigma\Delta[n]\) is inner anodyne as follows: First, we may form the pushout product of the spine inclusion \(\iota_n:\Sp[n]\hookrightarrow \Delta[n]\) with the monomorphism \(b:\partial\Delta[1]\hookrightarrow \Delta[1]\).  Since inner anodyne maps are closed under pushout-products, this gives us an inner-anodyne map \[\iota_n \wedge b: \Delta[1]\times \Sp[n] \cup \partial\Delta[1]\times \Delta[n]\hookrightarrow \Delta[n]\times \Delta[1].\]  

However, the source of this map admits another canonical map induced by the commutativity of the square under projection to the suspension of the spine, which is bipointed.  That is to say, we are looking at the canonical map \[m:\Delta[1]\times \Sp[n] \cup \partial\Delta[1]\times \Delta[n] \to \Sigma \Sp[n].\]  Pushing out the pushout-product map along this map \(m\), we obtain an inner anodyne map from the suspension of the spine to the suspension of the \(n\)-simplex.  To see that the inclusion of any nontrivial edge into the suspension of the spine is inner anodyne, suppose we're given a lifting diagram with the inclusion of a nontrivial edge into the suspension of the spine on the left together with an inner fibration against which we must find a lift. However, since the suspension of the spine is just a finite-length family of 2-disks glued together along their opposite edges, we may continually extend the original edge along degenerate edges in front or behind and thereby find a lift of each disk by induction.

Then in particular, the maps \(\Sigma\Delta[n]\to \Delta[1]\) are retracts of the inner anodyne inclusions \[\Delta[1]\hookrightarrow \Sigma\Sp[n]\hookrightarrow \Sigma \Delta[n]\],  and therefore the functors \((-)[A]\) must send them to weak equivalences, since those functors necessarily preserve inner anodyne maps, as they send spine inclusions to Segal core inclusions.   

It suffices then to show that for any presheaf \(A\) on \(\C\), the cosimplicial object defined by the functor \(\Sigma(-)[A]\) is Reedy cofibrant both as an ordinary resolution and as a cosimplicial resolution in the coslice under \(\Delta_1[\emptyset]\).  The second case is immediate, since the coslice version of the cosimplicial resolution preserves monomorphisms and colimits.  Then we consider the other case.

However, this case is similarly trivial because the functor preserves connected colimits of simplicial sets and monomorphisms, so, in particular, the image of the boundary of the \(n\)-simplex injects into the \(n\)th component, which gives that the latching map is a monomorphism, and therefore that the object is a Reedy cofibrant cosimplicial object.  
\end{proof}
\begin{cor}For any \(\mathsf{W}_{\Sc}\)-fibrant \(\psh{\C}\)-enriched simplicial set \(X\) equipped with two vertices \(x_0,x_1\), the simplicial set \(\Map_X(x_0,x_1)(A)\) models the homotopy function complex whose set of connected components is \(\psh{\C}[\mathsf{W}_{\Sc}^{-1}]((\Delta_1[A],0,1),(X,x_0,x_1))\). Similarly, when \(X\) is \(\mathsf{W}_{\Sc}\)-fibrant, the simplicial set \(E(A,X)_n=\Hom(\Sigma(\Delta_n)[A],X)\) gives a model for the homotopy function complex \(H_{\mathsf{W}_{\Sc}}(\Delta_1[A],X)\).  
\end{cor}
\begin{proof}This follows immediately from the preceding lemma, since we have merely constructed homotopy function complexes from the given resolutions.  
\end{proof}

\subsection{The $\Delta\wr \C$-localizer $\mathsf{W}_{\wr}$}\label{weakenrichment}
We fix an accessible cartesian regular \(\C\)-localizer \(\W\).  We will give the definition of the \(\Delta\wr C\)-localizer \(\W_\wr\), and using \cite{rezk-theta-n-spaces}*{Theorem 8.1}, we will show that it is cartesian.  Moreover, we will show that if \(S\) is a class of strong generators for \(\W\), then \(\W_{\wr}\) is strongly generated (over \(\W_\Sc\)) by \(\Delta_1[S]\), the class comprising those maps \(\Delta_1[f]:\Delta_1[A]\to \Delta_1[B]\) such that \(f:A\to B\) belongs to \(S\).    
\begin{defn}We define a \dfn{suspended \(\W\)-equivalence} to be a map of the form \(\Delta_1[f]:\Delta_1[A]\to \Delta_1[B]\), where \(f:A\to B\) belongs to \(\W\).  
\end{defn}
\begin{defn} We define \(\mathsf{W}_\wr\) to be the regular completion of the \(\Delta\wr\C\)-localizer generated by the class comprising:
\begin{enumerate}
\item [(i)] The suspended \(\mathsf{W}\)-equivalences.
\item [(ii)] The Segal core inclusions \(\Sc_n[c]\hookrightarrow \Delta_n[c]\) for any family of objects \(c=(c_1,\dots,c_n)\) of \(\C\).
\end{enumerate}     
\end{defn}
Before we begin, we first fix some notation, to avoid confusion.  We will denote the \emph{regular} \(\Delta\wr \C\times \C\)-localizer generated by a class of maps \(S\times \Delta_0\) with \(S\) a class of maps in \(\psh{\Delta\wr \C\times \Delta}\) by \(\sW(S\cup \W_\infty)\), which coincides with the localizer \(\W(S\cup \mathsf{R}(\W_\infty))\).  If \(S\) is a class of maps in \(\psh{\Delta\wr \C}\), we will, by abuse of notation, let \(\sW(S)=\sW(S\times \Delta_0)\).  
\begin{thm}[\cite{rezk-theta-n-spaces}*{Proposition 8.5}] The regular \(\Delta\wr \C\times \Delta\)-localizer \(\sW(\Sc\cup \Delta_1[\W])\) is cartesian.  Moreover, if \(S\) is a class of strong regular generators for \(\W\), then \(\sW(\Sc\cup \Delta_1[\W])=\sW(\Sc\cup \Delta_1[S])\)
\end{thm}
\begin{proof}See \cite{rezk-theta-n-spaces}*{Proposition 8.2-8.5}.
\end{proof}
This immediately gives us the corollary
\begin{cor}The localizer \(\W_\wr\) is cartesian, and if \(S\) is a class of strong regular generators of \(\W\), then \(\Delta_1[S]\cup \Sc \cup \{j\}\) is a class of strong regular generators for \(\W_\wr\).   
\end{cor}





\section{The theory of $\G$-extensions and strict $\omega$-categories}

This section is mainly meant to be a quick review of the main results in the second and third chapters of \cite{ara-thesis}, and the author makes no claims of originality in this section. 

\subsection{The globe category $\G$}

A good deal of this section is taken straight from the first chapter of Dimitri Ara's thesis, \cite{ara-thesis}.

Let \(\mathbb{G}_n\) denote the category presented as the free category on 
\begin{equation*}
\begin{tikzpicture}
\matrix (m) [matrix of math nodes, row sep=3em,
column sep=3em, text height=1.5ex, text depth=0.25ex]
{ D_0 & D_1 & \dots &D_{n-1}& D_n \\};
\path[->, font=\scriptsize]
(m-1-1) edge[transform canvas={yshift=0.2em}]   node[auto]{\(\scriptstyle \sigma_1\)} (m-1-2)
(m-1-1) edge[transform canvas={yshift=-0.2em}]   node[auto, swap]{\(\scriptstyle \tau_1\)} (m-1-2)
(m-1-2) edge[transform canvas={yshift=0.2em}]  node[auto]{\(\scriptstyle \sigma_2\)}(m-1-3)
(m-1-2)	edge[transform canvas={yshift=-0.2em}]  node[auto, swap]{\(\scriptstyle \tau_2\)} (m-1-3)
(m-1-3) edge[transform canvas={yshift=0.2em}]  node[auto]{\(\scriptstyle \sigma_{n-1}\)}(m-1-4)
(m-1-3)	edge[transform canvas={yshift=-0.2em}] node[auto, swap]{\(\scriptstyle \tau_{n-1}\)} (m-1-4)
(m-1-4) edge[transform canvas={yshift=0.2em}]  node[auto]{\(\scriptstyle \sigma_{n}\)}(m-1-5)
(m-1-4)	edge[transform canvas={yshift=-0.2em}] node[auto, swap]{\(\scriptstyle \tau_{n}\)} (m-1-5);
\end{tikzpicture}
\end{equation*}
modulo the the coglobular relations, \[\sigma_{i+1}\sigma_i=\tau_{i+1}\sigma_i \qquad \text{and} \qquad \tau_{i+1}\tau_i=\sigma_{i+1}\tau_i\] for \(1\leq i\leq n\).  There is an obvious inclusion map \(\mathbb{G}_n\hookrightarrow \mathbb{G}_{n+1}\) for each \(n \in \mathbf{N}\).  This defines a directed system, and we denote its colimit in \(\cat{Cat}\) by \(\mathbb{G}\).  

For integers \(0\leq i\leq j\), we define maps \(D_i\to D_j\) in \(\mathbb{G}\): 

\[\sigma^j_i=\sigma_j\hdots\sigma_{i+1}\qquad \text{and}\qquad\tau^j_i=\tau_j\hdots\tau_{i+1}\]

It follows by induction and the coglobular relations that given \(D_n, D_m\in \mathbb{G}\), we have that 
\[\Hom_\mathbb{G}(D_n,D_m)=
\begin{cases}\{\sigma_n^m,\tau_n^m\} & \text{if \(n<m\)}\\
\{\id_{D_n}\} & \text{if \(n=m\)}\\
\emptyset & \text{otherwise}\end{cases} \]

For any presheaf in \(X\in \operatorname{Ob}\hat{\G}=\cat{Cat}(\G^\op,\cat{Set})\), by abuse of notation, we let \(s_n=X_{\sigma_n}\) and \(t_n=X_{\tau_n}\).

\subsection{Globular patterns and $\G$-extensions}

For \(k\geq 2\), we define the category \(I_k\) to be the category associated with the ordered set \(\{(i,j): 0\leq i \leq 1 \wedge 0\leq j\leq k \wedge (i,j)\neq (0,k)\}\) ordered by the relation that \((i,j)\leq (i',j')\) if and only if \(i'-i=1\) and \(0\leq j'-j\leq 1\).  When \(k=1\), we let \(I_k=\ast\). 

\begin{defn}
A functor \(\eta:I_k\to \mathbb{G}\) for \(k\geq 1\) is called a \dfn{globular pattern} when the following conditions are satisfied:

\begin{enumerate}
\item [(i)] Every morphism of the form \(\alpha:(0,j)\to (1,j)\) in \(I_k\), the map \(\eta(\alpha)=\sigma_n^m\) for some \(m>n\geq 0\)
\item [(ii)] Every morphism of the form \(\beta:(0,j)\to (1,j+1)\) in \(I_k\), the map \(\eta(\beta)=\tau_n^m\) for some \(m>n\geq 0\)
\end{enumerate}

If \((C,F:\G\to C)\) is a category under \(\G\), a functor \(\eta:I_k\to C\) for some \(k\geq 1\) such that \(\eta\) factors as \(\eta_0^\ast F=F\circ \eta_0\) for some globular pattern \(\eta_0:I_k\to \G\) is called a \dfn{globular pattern in \((C,F)\)}.

If \((C,F:\G^\op\to C)\) is a category under \(\G^\op\), we define a \dfn{coglobular pattern in \((C,F)\)} to be a functor \(\eta:I_k^\op \to C\) for some \(k\geq 1\) such that the corresponding functor \(\eta^{op}:I_k\to C^\op\) is a globular pattern in \((C^\op,F^\op:\G\to C^\op)\).

We define globular sums (resp. globular products) in a category \((C,F:\G\to C)\) under \(\G\) (resp. in a category \((C,F:\G^\op\to C)\) under \(\G^\op\)), to be colimits (resp. limits) of globular patterns (resp. coglobular pattern) \(\eta\) in \((C,F)\).  
\end{defn} 

\begin{defn}
We say that a category \((C,F:\G\to C)\) under \(\G\) (resp. in a category \((C,F:\G^\op\to C)\) under \(\G^\op\)) to be a \dfn{globular \(\G\)-extension} (resp. \dfn{ globular \(\G\)-coextension}) if it contains all globular sums (resp. globular products).  A morphism of \(\G\)-extensions is a functor under \(\G\) that preserves all  globular sums.  Unless otherwise noted, we will simply refer to these as \(\G\)-extensions and \(\G\)-coextensions respectively.
\end{defn}
 
\begin{defn}
Given a \(\G\)-extension \((C,F)\) and a category \(D\), we define a \dfn{\(D\)-valued \(C\)-model} to be a functor \(G:C^\op\to D\) such that \((D,G\circ F^\op:\G^\op\to D)\) is a \(\G\)-coextension and such that the functor \(G^{op}:C\to D^\op\) is a morphism of \(\G\)-extensions.  We define the category \(\Mod(C,D)\) to be the full subcategory of \(D^{C^\op}\) spanned by the \(D\)-valued \(C\)-models.  

By abuse of notation, for any category \(D\) we denote the full subcategory of \(D^{\G^\op}\) spanned by the \(\G\)-coextensions by \(\Mod(\G,D)\).  We call the objects of this category \dfn{globular sets taking values in \(D\)}.
\end{defn}

\subsection{The initial $\G$-extension $\Theta_0$}

\begin{prop}
There exists a unique \(\G\)-extension \((\Theta_0,\iota_0\:\G\to \Theta_0)\) such that the induced transformation 
\[\iota_0^*:\Mod(\Theta_0,\cdot)\to \Mod(\G,\cdot)\]
is an equivalence of 2-functors.  Moreover, for any \(\G\)-extension \(D,F\), there exists a unique (up to isomorphism) factorization of the structure map \(F\) as the composite of the map \(\iota_0:\G\to \Theta_0\) with some morphism of \(\G\)-extensions \(F_0:\Theta_0\to D\) .  
\end{prop}
\begin{proof}
We take \(\Theta_0\) to be the full subcategory of \(\widehat{\G}\) spanned by the globular sums.  This gives a \(\G\)-extension because \(\Theta_0\subseteq \widehat{\G}\) contains the image of the Yoneda embedding, which gives us a factorization \(\G\hookrightarrow \widehat{\G}=\G\overset{\iota_0}{\hookrightarrow} \Theta_0\overset{\gamma_0}{\hookrightarrow}\widehat{\G}\).   

Then we would like to construct an inverse for the transformation \(\iota_0^\ast\). Let \(\cat{Comp}\) be the 2-subcategory of \(\cat{Cat}\) spanned by the complete categories with limit-preserving functors between them.  By the universal property of the co-Yoneda embedding, we have that for any complete category \(B\), \(\cat{Cat}(\G^\op,B)\simeq \cat{Comp}(\widehat{\G}^{op},B)\), naturally in \(B\). Also, since every complete category \(B\) necessarily contains all globular products for any functor \(\G^\op\to B\), and since every continuous functor \(X:\widehat{G}^{op}\to B\) necessarily preserves all globular products \emph{a fortiori}, we have an embedding \(\cat{Comp}(\widehat{\G}^{op},B)\hookrightarrow \Mod(\widehat{\G},B)\)  

Then we see that we have a chain of transformations natural in \(C\), the composite of which we will call \(F_C\),
\[\Mod(\G,C)\overset{\iota}{\hookrightarrow} \cat{Cat}(\G^\op,\widehat{C}) \simeq \cat{Comp}(\widehat{G}^\op,\widehat{C})\hookrightarrow\Mod(\widehat{\G},\widehat{C})\overset{\gamma_0^\ast}{\to} \Mod(\Theta_0,\widehat{C}),\]
but for each globular set \(X\) taking values in \(C\), the object \(FX:\Theta_0\to \widehat{C}\) necessarily factors uniquely (up to specified isomorphism) through the Yoneda embedding, since the category \(C,i_0^\ast FX:\G\to C\) under \(\G\) contains all globular products, and the Yoneda embedding \(C\hookrightarrow \widehat{C}\) necessarily preserves them.  

This means, in particular, that \(F_C\) factors through the inclusion \[\Mod(\Theta_0,C)\hookrightarrow \Mod(\Theta_0,\widehat{C}).\]  We let \(H_C\) denote the factor of \(F_C\) going from \(\Mod(\G,C)\) to \(\Mod(\Theta_0,C)\) (naturally in \(C\).  Suppressing the \(C\), we see that \(H\) clearly inverse to \(\iota_0^\ast\) by way of the factorization.   The fact that \(\Theta_0\) is initial in the category of \(\G\)-extensions follows from the earlier claim by letting \(C=D^\op\).
\end{proof}

We recall a proposition of Ross Street in \cite{street}:

\begin{prop} There exists an order structure \(\bltri_A\) on \(\overcat{\G}{A}\) for any presheaf \(A\) on \(\G\) such that maps \(A\to B\) of presheaves induce order preserving maps \(\overcat{\G}{A}\to \overcat{\G}{B}\) and such that the associated ordered sets \((\overcat{\G}{X}, \bltri_X)\) of the objects \(X\) of \(\Theta_0\subset \widehat{\G}\) are finite and linearly ordered.
\end{prop}
\begin{proof}
We construct a functorial order structure on \(\El(A)=\operatorname{Ob}(\overcat{\G}{A})\) for a presheaf \(A\) on \(\G\) following Street in \cite{street} by taking the reflexive transitive closure \(\blacktriangleleft_X\) of the relation\(\prec_X\) defined such that given a map \(\alpha:D_n\to X\), 
\begin{equation}\label{streetord} \alpha \prec_X \beta \qquad \text{if and only if \(\alpha=s_{n+1}(\beta)\) or \(t_n(\alpha)=\beta\)}.\end{equation}\
To show that this order structure is functorial, it suffices to show that the function \(\El(f):\El(A)\to \El(B))\) preserves the order structure generated by the relation above.  However, this follows from the definition \eqref{streetord} and the commutativity of the induced functor with the source and target maps. 

We will show that given any globular pattern \(H:I_k \to \G\) with colimit \(X\) in \(\widehat{\G}\), the ordered set \((\El(X),\bltri_X)\) is linearly ordered.  We proceed by induction as follows: Assume that for every globular pattern \(I_j\to \G\) with \(j<k\), the claim holds for \(colim I_j\).  Then we note that we may decompose \(X\cong Y\coprod_{D_{i'_{k-1}}} D_{i_k} \), where \(Y\) is the colimit of the globular pattern \(\iota_0^*H:I_{k-1}\to \G\), where \(\iota_0:I_{k-1}\to I_k\) is the obvious inclusion on the first \(k-1\) components.  Note that this gives a canonical factorization \(D_{i'_{k-1}}\to Y\) as \(D_{i'_{k-1}} \to D_{i_{k-1}} \to Y\), where the first map is \(\sigma^{i_{k-1}}_{i'_{k-1}}\).  

Then if \(\gamma:D_{i_\gamma}\to X, \lambda: D_{i_\lambda}\to X\) are both maps factoring through either \(\alpha_k:D_{i_k}\to X\) or \(\alpha_Y:Y\to X\), then \(\gamma \bltri_X \lambda\) reduces to \(\gamma \bltri_{D_{i_{k}}}\lambda\) or \(\gamma\bltri_Y \lambda\).  So without loss of generality, since every element of \(\El(X)\) factors through at least one globular summand, we may assume that \(\gamma\) belongs to \(\El(D_{i_k}) - \im(\El(\tau^{i_k}_{i'_{k-1}}))\) and that \(\lambda\) belongs to \(\El(Y)-\im(\El(\sigma^{i_{k-1}}_{i'_{k-1}}))\).  

First, notice that for \(j\leq i'_{k-1}\), we have that \(s^{i'_{k-1}}_j(\alpha'_{k-1})=s^{i_{k-1}}_j(\alpha_{k-1})\) and that \(t^{i'_{k-1}}_j(\alpha'_1=t^{i_k}(\alpha_Y)\).  

Since \(\gamma\) belongs to \(\El(D_{i_k}) - \im(\El(\tau^{i_k}_{i'_{k-1}}))\), we have that \(\gamma\) lives in the strict \(\bltri_{D_{i_k}}\)-interval \((s^{i_k}_{i'_{k-1}}(\alpha_k),t^{i_k}_{i'_{k-1}}(\alpha_k))_{\bltri_{D_{i_1}}})\).  Similarly, since \(\lambda\) belongs to \(\El(Y)-im(\El(\sigma^{i_{k-1}}_{i'_{k-1}}))\), this in particular implies that \(s^{i_{k-1}}_{i'_{k-1}}(\alpha_{k-1}) \bltri_Y \lambda\).  

Since \[\gamma \bltri_{D_{i_k}} t^{i_k}_{i'_{k-1}}(\alpha_k),\] \[t^{i_k}_{i'_{k-1}}(\alpha_k)=\alpha'_{k-1}= s^{i_{k-1}}_{i'_{k-1}}(\alpha_{k-1}),\] and also \[s^{i_{k-1}}_{i'_{k-1}}(\alpha_{k-1})\bltri_Y \lambda,\] we have that \(\gamma \bltri \lambda\).  It is clear that antisymmetry holds, since in a case such as the one above, there is exactly one traversable path between components, and when they are in the same component, antisymmetry is inherited from the lower-order cases by induction. 
\end{proof} 

\begin{prop}
The category \(\Theta_0\) is the full subcategory of \(\psh{\G}\) spanned by the objects \(X\) such that the ordered set \[(\overcat{\G}{X}, \bltri_X)\] is finite and linearly ordered.
\end{prop}
\begin{proof}
Since \(\overcat{\G}{X}\) is finite, let \(i_1,\dots i_n\) be the heights corresponding to the elements of maximal height and ordered as a subset of the total order under \(\bltri_X\).  (Finish proof later)
\end{proof}

\begin{prop}
Every morphism \(f:A\to B\)  in \(\Theta_0\) is monic.
\end{prop}
\begin{proof} It suffices to show that the natural transformation between presheaves on \(\G\) associated with the map \(f\) is objectwise injective.  To see this, we give \(A_n\) the structure of a linear digraph, where \(\alpha \prec_n \beta\) if and only if \(\alpha \bltri 
\beta\) and \((\alpha,\beta)_\bltri\cap A_n=\emptyset\), where \((\alpha,\beta)\) denotes the strict open interval between \(\alpha\) and \(\beta\). We see that \(f_n\) preserves \(prec_n\), but \(prec_n\) is irreflexive, so in particular \(f_n\) is injective.    

\end{proof}

\subsection{The globular envelope of a category under $\Theta_0$}

Let \((C,D_C)\) be a \(\G\)-extension.  Then we say that a functor \(F:C\to E\) is a \dfn{globular \((C,D_C)\)-extension} if \((D,D_C^\ast(F))\) is a \(\G\)-extension and \(F\) is a morphism of \(\G\)-extensions.  We define the \dfn{category of globular \((C,D_C)\)-extensions}, denoted \((C,D_C)\cat{-Ext}\) to be the category whose objects are globular \((C,D_C)\)-extensions and whose arrows are morphisms of \(\G\)-extensions under \((C,D_C)\).  Unless otherwise noted, we will abuse notation and simply denote this category simply by \(C\cat{-Ext}\), with its objects similarly called \(C\)-extensions.  

\begin{prop} The category of \(\Theta_0\)-extensions is equivalent to the category of \(\G\)-extensions.  \end{prop}
\begin{proof} Immediate from the definitions.
\end{proof}

\begin{prop} If \((C,D_C)\) is a \(\G\)-extension, any functor under \(C\) between \(C\)-extensions is a morphism of \(C\)-extensions.
\end{prop}
\begin{proof} Let \[H:(X,F_X)\to(Y,F_Y)\] be a functor under \(C\) between \(C\)-extensions, and let 
\begin{align*}D_X&=D_C^\ast(F_X) \intertext{and} D_Y&=D_C^\ast(F_Y).\end{align*}  Then any globular sum in \((X,D_X)\) is the image under \(F_X\) of a globular sum in \((C,D_C)\). Then since \(HF_X=F_Y\), and \(F_Y\) sends globular sums in \((C,D_C)\) to globular sums in \((Y,D_Y)\), it follows that any globular sum in \((X,D_X)\) must map under \(H\) to a globular sum in \((Y,D_Y)\).  Therefore, \(H\) is a morphism of \(\G\)-extensions and belongs to \((C,D_C)\cat{-Ext}\).  
\end{proof}
This immediately yields the corollary:
\begin{cor} For any \(\G\) extension \((C,D_C)\), the category \(C\cat{-Ext}\) is a full subcategory of \(\overcat{C}{\cat{Cat}}\)\end{cor}.

Until the close of this subsection, we denote the forgetful functor \(\Theta_0\cat{-Ext} \to \overcat{\Theta_0}{\cat{Cat}}\) by the letter \(U\).

\begin{defn}Given a category \(C\) equipped with a functor \(\Theta_0\to C\), we say that a functor \(C\to U(C')\) under \(\Theta_0\) \dfn{exhibits \(C'\) as a globular envelope of \(C\)} if the following property holds: 

Given any solid arrow diagram
\begin{equation*}
\begin{tikzpicture}
\matrix (b) [matrix of math nodes, row sep=3em,
column sep=1.5em, text height=1.5ex, text depth=0.25ex]
{   & C & \\
  U(C') & & U(D) \\};
\path[->, font=\scriptsize]
(b-1-2) edge (b-2-1)
        edge (b-2-3)
(b-2-1) edge [dotted] node [auto,swap] {\(\scriptstyle U(f)\)} (b-2-3);
\end{tikzpicture},
\end{equation*}
there exists a unique arrow \(f:C'\to D\) in \(\Theta_0\cat{-Ext}\) such that \(U(f)\) gives the desired dotted arrow.  In such a situation, we will call \(C'\) a \dfn{globular envelope} for \(C\).  It is clear from the definition that any two globular envelopes for \(C\) are unique up to unique isomorphism.  
\end{defn}

\begin{thm}[\cite{ara-thesis}*{2.6}] Every small category under \(\Theta_0\) admits a globular envelope.
\end{thm}
\begin{proof} See \cite{ara-thesis}*{2.6}.   
\end{proof}

\subsection{Categorical $\G$-extensions}

A categorical \(\G\)-extension should, to a first approximation, be a \(\G\)-extension \(F:\G\to C\) together with the data of co-composition and co-degeneracy morphisms endowing the corresponding globular classified by the \(\G\)-coextension \[F^\op:\G^\op \to C^\op\] with the structure of a strict \(\omega\)-category internal to \(C^{op}\).  We write out what this means explicitly:

\begin{defn}
For \(i\geq j\geq 0\)
A \dfn{precategorical \(\G\)-extension} is specified by the following data:
\begin{enumerate}
\item [(i)] A functor \(D:\G \to C\) equipping \(C\) with the structure of a \(\G\)-extension.  We write \(D(D_n)\) simply as \(D_n\), and for \(f\in \G\), simply write \(D(f)\) as \(f\).  
\item [(ii)] For each \(i>j\geq 0\), a morphism \(\nabla^i_j:D_i \to D_i \coprod_{D_j} D_i\)
\item [(iii)] For each \(i\geq 0\), a morphism \(\kappa_i:D_{i+1}\to D_i\)
\end{enumerate}
satisfying the following axioms:
\begin{enumerate}
\item [(PC1)] For each \(i> 0\), we have that \[\kappa_i \sigma_{i+1}=\id_{D_i} \qquad \text{and} \qquad   \kappa_i \tau_{i+1}=\id_{D_i}\]
\item [(PC2)] For each \(i>j\geq 0\), let \(\varepsilon_1\) and \(\varepsilon_2\) denote the two canonical maps \(D_i\to D_i\coprod_{D_j} D_i\), we have that: 
\begin{align*}
\nabla^i_j \sigma_i = &\begin{cases}\varepsilon_2\sigma_i & \text{if \(j=i-1\)}\\ (\sigma_i\coprod_{D_j} \sigma_i)\nabla^{i-1}_j &\text{otherwise}\end{cases}\\
\intertext{and}\\
\nabla^i_j \tau_i = &\begin{cases}\varepsilon_1\tau_i & \text{if \(j=i-1\)}\\ (\tau_i\coprod_{D_j} \tau_i)\nabla^{i-1}_j & \text{otherwise}\end{cases}
\end{align*}
\end{enumerate}

\end{defn}

In keeping with Ara's treatment, we will fix the notations \[\kappa^j_i=\kappa_j\hdots\kappa_{i-2}\kappa_{i-1}\qquad\text{and}\qquad \nabla_k=\nabla^k_{k-1},\]
for \(i\geq j\geq 0\) and \(k>0\), respectively.  

\begin{defn} Using the same notation as above, we say that a precategorical \(\G\)-extension is a \dfn{categorical \(\G\)-extension} if it satisfies the following axioms:
\begin{enumerate}
\item[(CC1)] Associativity:\\
For \(i>j\geq 0\), the diagram
\begin{equation*}
\begin{tikzpicture}
\matrix (ass) [matrix of math nodes, row sep=4em,
column sep=4em, text height=1.5ex, text depth=0.25ex]
{ D_i & D_i\coprod_{D_j}D_i \\
   D_i\coprod_{D_j}D_i &  D_i\coprod_{D_j}D_i\coprod_{D_j}D_i \\};
\path[->, font=\scriptsize]
(ass-1-1) edge node[auto]{\(\scriptstyle \nabla^i_j\)} (ass-1-2)
          edge node[auto,swap]{\(\scriptstyle \nabla^i_j\)} (ass-2-1)
(ass-2-1) edge node[auto]{\(\scriptstyle \id_{D_i}\coprod_{D_j}\nabla^i_j\)} (ass-2-2)
(ass-1-2) edge node[auto,swap]{\(\scriptstyle \nabla^i_j\coprod_{D_j} \id_{D_i}\)} (ass-2-2);
\end{tikzpicture}
\end{equation*}
commutes.
\item[(CC2)]Strict interchange:\\
For \(i>j>k\geq 0\), the diagram
\begin{equation*}
\begin{tikzpicture}[bij/.style={above,sloped,inner sep=0.5pt}]
\matrix (ich) [matrix of math nodes, row sep=2.5em,
column sep=1.3em, text height=1.5ex, text depth=0.25ex]
{
                  & D_i & \\
D_i\coprod_{D_k}D_i & & D_i\coprod_{D_j}D_i \\
(D_i\coprod_{D_j}D_i)\coprod_{D_k} (D_i\coprod_{D_j}D_i) & & (D_i\coprod_{D_k}D_i)\coprod_{D_j\coprod_{D_k}D_j} (D_i\coprod_{D_k}D_i) \\};
\path[->, font=\scriptsize]
(ich-1-2.240) edge node[auto,swap]{\(\scriptstyle \nabla^i_k\)} (ich-2-1.60)
(ich-1-2.300) edge node[auto]{\(\scriptstyle \nabla^i_j\)} (ich-2-3.120)
(ich-2-1) edge node[auto]{\(\scriptstyle \nabla^i_j\coprod_{D_k}\nabla^i_j\)} (ich-3-1)
(ich-2-3) edge node[auto]{\(\scriptstyle \nabla^i_k\coprod_{\nabla^j_k}\nabla^i_k\)} (ich-3-3)
(ich-3-1) edge node[bij] {\(\sim\)}
							 node[below] {\(\scriptstyle \Phi\)} (ich-3-3);
\end{tikzpicture}
\end{equation*}
commutes, where \(\Phi\) is the unique isomorphism between the objects \[(D_i\coprod_{D_j}D_i)\coprod_{D_k}(D_i\coprod_{D_j}D_i)\] and \[(D_i\coprod_{D_k}D_i)\coprod_{D_j\coprod_{D_k}D_j} (D_i\coprod_{D_k}D_i)\] viewed as cones on the diagram
\begin{equation*}
\begin{tikzpicture}
\matrix (ijk) [matrix of math nodes, row sep=2em,
column sep=2em, text height=1.5ex, text depth=0.25ex]
{
D_i& D_k & D_i \\
D_j & D_k & D_j \\
D_i & D_k & D_j \\};
\path[->, font=\scriptsize]
(ijk-1-2) edge node[auto,swap]{\(\scriptstyle \sigma_k^i\)} (ijk-1-1)
					edge node[auto]{\(\scriptstyle \tau_k^i\)} (ijk-1-3)
(ijk-2-1) edge node[auto]{\(\scriptstyle \sigma_j^i\)} (ijk-1-1)
					edge node[auto,swap]{\(\scriptstyle \tau_j^i\)} (ijk-3-1)
(ijk-3-2) edge node[auto]{\(\scriptstyle \sigma_k^i\)} (ijk-3-1)
					edge node[auto,swap]{\(\scriptstyle \tau_k^i\)} (ijk-3-3)
(ijk-2-3) edge node[auto,swap]{\(\scriptstyle \sigma_j^i\)} (ijk-1-3)
					edge node[auto]{\(\scriptstyle \tau_j^i\)} (ijk-3-3)
(ijk-2-2) edge node[auto,swap]{\(\scriptstyle \sigma_k^j\)} (ijk-2-1)
					edge node[auto]{\(\scriptstyle \tau_k^j\)} (ijk-2-3)
(ijk-2-2) edge[-,double equal sign distance] node[auto,swap]{\(\scriptstyle \id_{D_k}\)} (ijk-1-2)
					edge[-,double equal sign distance] node[auto]{\(\scriptstyle \id_{D_k}\)} (ijk-3-2);
\end{tikzpicture},
\end{equation*}
arising from the fact that both cones are initial (that is, both objects are colimits of the above diagram).
\item[(c)] Left and right unitality:\\
For \(i>j\geq 0\), the diagram
\begin{equation*}
\begin{tikzpicture}[bij/.style={above,sloped,inner sep=0.5pt}]
\matrix (idc) [matrix of math nodes, row sep=3em,
column sep=3em, text height=1.5ex, text depth=0.25ex]
{
                  & D_i & \\
D_i \coprod_{D_j} D_j & D_i \coprod_{D_j} D_i & D_j \coprod_{D_j} D_i \\};
\path[->, font=\scriptsize]
(idc-1-2) edge node[auto]{\(\scriptstyle \nabla^i_j\)} (idc-2-2)
(idc-1-2) edge node[bij]{\(\scriptstyle \sim\)} (idc-2-1)
(idc-1-2) edge node[bij]{\(\scriptstyle \sim\)} (idc-2-3)
(idc-2-2) edge node[auto]{\(\scriptstyle \id_{D_i}\coprod_{D_j}\kappa^j_i\)} (idc-2-1)
(idc-2-2) edge node[auto,swap]{\(\scriptstyle \kappa^j_i\coprod_{D_j} \id_{D_i}\)} (idc-2-3);
\end{tikzpicture},
\end{equation*}
commutes.
\item[(d)] Functoriality of units:\\
For \(i>j\geq 0\), the diagram
\begin{equation*}
\begin{tikzpicture}
\matrix (funct) [matrix of math nodes, row sep=3em,
column sep=3em, text height=1.5ex, text depth=0.25ex]
{ D_{i+1} & D_{i+1}\coprod_{D_j} d_{i+1} \\
   D_i & D_{i+1}\coprod_{D_j}\\};
\path[->, font=\scriptsize]
(funct-1-1) edge node[auto]{\(\scriptstyle \nabla^{i+1}_j\)}(funct-1-2)
        edge node[auto,swap]{\(\scriptstyle \kappa_i\)} (funct-2-1)
(funct-2-1) edge node[auto,swap]{\(\scriptstyle \nabla^i_j\)}(funct-2-2)
(funct-1-2) edge node[auto]{\(\scriptstyle \kappa_i\coprod_{D_j} \kappa_i\)} (funct-2-2);
\end{tikzpicture}
\end{equation*}
\end{enumerate}
\end{defn}

\begin{defn}
A morphism  of (pre)categorical \(\G\)-extensions is defined to be a morphism of the underlying \(\G\)-extensions preserving the cocategorical operations \(\nabla^i_j\) and \(\kappa_i\).  
\end{defn}

\begin{defn} Given a category \(C\), a \dfn{strict \(\omega\)-(pre)category internal to \(C\)} is defined to be a globular set \(D^\op:\G^{op}\to C\) together with two specified families of operations \((\nabla^i_j:D_i\to D_i\coprod_{D_j} D_i){i>j\geq 0}\) and \((kappa_i:D_i\to D_{i-1})_{i>0}\) on \(C^\op\) such that the triple \((D,(\nabla^i_j)_{i>j\geq 0},(\kappa_i)_{i>0})\) gives \(C^{op}\) the structure of a (pre)categorical extension.  

By abuse of notation, we will, when the meaning is clear, simply refer to such a triple by its underlying globular set. A morphism \(X\to Y\) of strict \(\omega\)-(pre)categories internal to \(C\) is defined to be a natural transformation \(X\to Y\) such that the induced map \(Y^\op \to X^\op\) respects the operations \(\nabla^i_j\) and \(\kappa_i\) for all \(i>j\geq 0\).   We denote the category of strict \(\omega\)-categories internal to \(C\) by \(\cat{\omega\-Cat}(C)\), or when \(C=\cat{Set}\), simply by \(\cat{\omega-Cat}\).  
\end{defn}


 
\subsection{$\Theta$ as the initial categorical extension}

We will give a description of \(\Theta\) as a \(\G\)-extension and show that the models for \(\Theta\) are precisely the strict \(\omega\)-categories.  

The first construction, due to Ara in \cite{ara-thesis}, uses his theory of globular envelopes together with a two-step brute-force construction by presentation, which we will recount here:

\begin{enumerate}
\item[(1)] Let \(\Theta_{\operatorname{pcat}}\) denote the globular envelope of the category obtained by formally adjoining the operations \(\nabla^i_j\) and \(\kappa_i\) and taking the quotient by the relations (PC1) and (PC2).
\item[(2)] Let \(\Theta\) denote the globular envelope of the category obtained from \(\Thetap\) by formally identifying the legs of the commutative diagrams in (CC1-CC4).
\end{enumerate}

Then we have the following result following immediately from the universal properties of  and the fact that \(\Theta\) and \(\Thetap\) are, by construction, a categorical extension and a precategorical extension respectively:

\begin{prop}
The canonical functor, natural in \(D\), \[\Mod(\Theta,D)\to \cat{\omega-Cat(D)}\] (respectively, \[\Mod(\Thetap,D)\to \cat{\omega-PCat(D)},\] also natural in \(D\)) is a natural equivalence of categories.
\end{prop}

\subsection{The combinatorial properties of $\Theta$}
We will obtain an explicit definition of the morphisms and objects in \(\Theta\) using the definition from the last section.  First, notice that we have a canonical morphism of \(\G\)-extensions \(\Theta_0\to \Theta\).  We would like to describe this functor using only the axioms for categorical extensions and the globular extension property of \(\Theta_0\).  Then it suffices to describe the hom-sets \(\Hom_\Theta(D_n,S)\) for \(n\geq 0\) and \(S\) any object of \(\Theta\).   We will call a morphism \(f:S\to T\) in \(\Theta\) a \dfn{spinal} monomorphism if it is the image of a morphism in \(\Theta_0\).

Let \(T\) be the object of \(\Theta_0\) (and \(\Theta\)) defined by a globular pattern 
\[D_{i_1}\leftarrow D_{i'_1}\to \dots \leftarrow D_{i'_{k-1}} \to D_{i_k},\] 
and let \(n\geq \heit(T)=\max_{1\leq j\leq k}(i_j)\).  Then there exists a canonical map \(\beta^T_n : \eta_{n(T)} \to T\) defined by the globular pattern \[D_{n}\leftarrow D_{i'_1}\to \dots \leftarrow D_{i'_{k-1}} \to D_{n}\] given by the iterated amalgamation of the appropriate maps \(\kappa^{i_j}_n: D_n \to D_{i_j}\), which we may depict schematically as \[\kappa^{i_1}_n\leftarrow D_{i'_1}\to \dots \leftarrow D_{i'_{k-1}} \to \kappa^{i_k}_n: .\]  Also, suppose we are given a family of nonnegative integers \(i'_j :0\leq j\leq k-1\) and a nonnegative integer \(n\geq i'_j\) for all \(0\leq j\leq k-1\) .  

Then we define the cocomposition operation for an object defined by a globular pattern of the form \[D_{n}\leftarrow D_{i'_1}\to \dots \leftarrow D_{i'_{k-1}} \to D_{n}\] by induction with respect to the special case where \(k=3\).  That is, we define \[\nabla^n_{i'_1,i'_2}:D_n\to D_n \coprod_{D_{i'_1}} D_n \coprod_{D_{i'_2}} D_n\] to be the composite of either leg in the diagram
\begin{equation*}
\begin{tikzpicture}
\matrix (b) [matrix of math nodes, row sep=3em,
column sep=3em, text height=1.5ex, text depth=0.25ex]
{ D_n & D_n \coprod_{D_{i'_1}} D_n \\
D_n\coprod_{D_{i'_2}} D_n &  D_n \coprod_{D_{i'_1}} D_n \coprod_{D_{i'_2}} D_n \\};
\path[->, font=\scriptsize]
(b-1-1) edge node[auto]{\(\scriptstyle \nabla^n_{i'_1}\)} (b-1-2)
        edge node[auto]{\(\scriptstyle \nabla^n_{i'_2}\)} (b-2-1)
(b-2-1) edge node[auto]{\(\scriptstyle \nabla^n_{i'_1}\coprod_{D_{i'_2}} \id_{D_n}\)} (b-2-2)
(b-1-2) edge node[auto]{\(\scriptstyle \id_{D_n} \coprod_{D_{i'_1}}  \nabla^n_{i'_2}\)} (b-2-2);
\end{tikzpicture}.
\end{equation*}    
In fact, the diagram above does commute, which follows from the interchange and unit axioms.  This gives us maps \(\nabla^n_{i'_1,\dots, i'_{k-1}}\) for any choices of \(i'_j\) such that \(i'_j \leq n\) for each \(j\).  Given any object \(T\) of \(\Theta_0\) with height at most \(n\geq 0\), we then define \(c_n^T\) to be the composite map \(D_n\to T\) given by the composite \(\beta^T_n\circ \nabla^n_{i'_1,\dots i'_{k_T}} : D_n\to \eta_n(T) \to T\), and we call it an \dfn{\(n\)-cospine of \(T\)}.  If \(n=\heit(T)\), we call the unique \(n\)-cospine the \dfn{principal cospine of \(T\)}.  We will say that a morphism \(f:S\to T\) in \(\Theta\) is \dfn{cospinal} if \(f\circ c_n^S\) is a cospine for \(T\).

Since the objects of \(\Theta\) represent functors that are \(\omega\)-categories internal to \(\cat{Set}\) (which follows from the fact that \(\Hom_{\Theta}(\cdot,S)\) sends colimits to limits), we see that the principal cospine of an object \(T\) gives the \(T\)-shaped composition map \(\Hom_{\Theta}(T,S)\to \Hom_{\Theta}(D_{\heit(T)},S)\), which gives us the ``total composite cell'' of \(S\) when \(T=S\).  Then the set of \(n\)-globes of the underlying \(\Theta_0\)-model of an object \(S\) in \(\Theta\) is precisely given by the set of pairs consisting of a cospine \(D_n\to T\) together with a spinal monomorphism \(T\to S\), since every globe of \(S\) is uniquely a composite of a subspine \(T\hookrightarrow S\).  That is, \(Hom_{\Theta}(D_n,S)=\coprod_{\heit(T)\leq n}\Hom_{\Theta_0}(T,S)\).  

Since every object \(R\) of \(\Theta\) is uniquely a globular sum of globes \(D_n\), we find that  \[\Hom_{\Theta}(R,S)=\varprojlim_{I_k}\Hom_{\Theta}(D_{n_j},S),\]
which gives us an explicit definition of the hom-sets in \(\Theta\).   

The above discussion implies easily that the following proposition holds:
\begin{prop}\label{spinalfactor} Every morphism \(S\to T\) in \(\Theta\) admits a unique decomposition into a cospinal map followed by a spinal monomorphism.
\end{prop}

We introduce a few definitions that we will use later:

\begin{defn} We call a map \(D_n\to T\) in \(\Theta\) a \dfn{sector} if it is a spinal monomorphism.  When such a map is maximal in the poset \(\overcat{\Theta_0}{T}\), we will call it an \dfn{input sector}.  
\end{defn}