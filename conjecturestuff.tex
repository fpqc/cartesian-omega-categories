This reduces the question to showing that if \(X\) is stably Segal and \(\iota\)-local, then so too is \(X^{D_i}\) for each \(i\geq 0\).  We will come back to this question after developing a bit of machinery.
\subsection{The homotopy $\omega$-category of a stably Segal simplicial cellular set}
We define a functor \(D\) from simplicial cellular sets \(\widehat{\Theta\times\Delta}\) to strict \(\omega\)-categories  by means of the following functor \(\Theta\times\Delta\to \cat{\omega}-cat\):  On objects, this is defined by \(([t],[n])\mapsto [t]\otimes E_n\), and on morphisms, it is defined in the obvious way (recall that \(E_n\) was earlier defined to be the standard \(\omega\)-polygraph resolution of \(J_n\)).  This functor induces an adjoint pair \[mathscr{P}\psh{\Theta\times \Delta}\rightleftarrows \cat{\omega-cat}:\mathscr{B}.\] 
\begin{thm} A map \(f:X\to Y\) of strict \(\omega\)-categories is a weak equivalence in the sense of \cite{lmw} if and only if \(\mathscr{B}(X)\to \mathscr{B}(Y)\) is an objectwise simplicial homotopy equivalence.
\end{thm}
\begin{proof}  
It follows from \cite{lmw} that \(\operatorname{RLax}(E_n,X)\) is a simplicial resolution of \(X\), so \(\mathscr{B}(X)_t=\Map([t],X)\), which implies that a weak equivalence of strict \(\omega\)-categories gives an objectwise simplicial weak equivalence.  

Conversely, suppose \(\mathscr{B}(X)\to \mathscr{B}(Y)\) is an objectwise simplicial weak equivalence.  Then for \(f:X\to Y\) to be a weak equivalence, it suffices to show that \(\Map(Z,X)\to \Map(Z,Y)\) is a weak equivalence for all strict \(\omega\)-categories \(A\).  However, for every strict \(\omega\)-category \(Z\), there exists a cofibrant strict \(\omega\)-category \(A\) equipped with a weak equivalence \(A\to Z\) such that \(\Map(A,-) \cong \Map(Z,-)\).   However, any cofibrant \(A\) can be obtained as the image of a polygraph.  

Since this is the case, we have that \(A\) is the colimit of the diagram of cofibrations  \(A^{(0)} \hookrightarrow A^{(1)} \hookrightarrow \dots\).  Then this is a homotopy colimit, and it suffices to show that the conclusion holds for each \(A^{(i)}\). The conclusion clearly holds for \(A^{(0)}\), which is a disjoint union of copies of \(D_0\).  Then suppose it holds for \(A^{(k)}\).  We know that \(A^{(k+1)}\) is the pushout of some diagram \[\coprod_{S_{k+1}} D_{k+1} \xleftarrow{\coprod_{S_{k+1}} \iota_{k+1}} \coprod_{S_{k+1}} \partial D_{k+1} \to A^{(k)}\], which is a homotopy colimit, since all of the objects are cofibrant, and since the lefthand map is a cofibration.  Then \(\Map(A^{(k+1)}, -)\) is the homotopy limit of the diagram \[\Map(\coprod_{S_{k+1}} D_{k+1},-) \to \Map(\coprod_{S_{k+1}} \partial D_{k+1},-) \leftarrow \Map(A^{(k)},-).\]  By induction, we have that \[\Map(A^{(k)},X) \to \Map(A^{(k)},Y)\] is a weak equivalence, and by our original hypothesis, we have that \[\Map(\coprod_{S_{k+1}} D_{k+1},X)\to \Map(\coprod_{S_{k+1}} D_{k+1},Y)\]  is a weak equivalence.  Then it suffices to show that \[\Map(\coprod_{S_{k+1}} \partial D_{k+1},X)\to \Map(\coprod_{S_{k+1}} \partial D_{k+1},Y)\] is a simplicial weak equivalence.  This is equivalent to showing that \[\Map(\partial D_{k+1},X)\to \Map(\partial D_{k+1}, Y)\] is a weak equivalence.  However, the boundary decomposes as the pushout \(D_k \leftarrow \partial D_k \to \partial D_k\), which is again a pushout of cofibrations, and therefore a homotopy pushout.  Then by induction and the hypothesis, we have that \[\Map(\partial D_{k+1},X) \to \Map(\partial D_{k+1}, Y)\] is a weak equivalence as well. Therefore, we find that \(\Map(A,X)\to \Map(A,Y)\) is a weak equivalence, which proves the claim.  
\end{proof}


\begin{defn} If \(X\) is a stably Segal object, we say that an \(n\)-cell in \(D_n\to X\) classifies an \dfn{equivalence} if its image in \(\mathscr{P}(X)\) is an \(\omega\)-equivalence between a parallel pair of \(n-1\)-cells.  Given a map \(A\to X\), we say that it is \dfn{\(n\)-local} if for every map \(D_m \to A\) with \(m>n\), the composite \(D_m \to A\to X\) classifies an equivalence.   For any simplicial cellular set \(A\), we let  \(H_\Delta(A,X)_{\Loc_n}\)  be the union of all of the path components in \(\HMap_\Delta(A,X)\) containing at least one \(n\)-local map \(A\to X\).  
\end{defn}

Recall Street's oriental functor \(\CMcal{O}:\Delta\to \cat{\omega-cat}\). We form the composite \[\Delta\xrightarrow{\CMcal{O}} \cat{\omega-cat} \xrightarrow{\operatorname{Collapse}} \cat{2-cat} \hookrightarrow \cat{\omega-cat} \xrightarrow{\mathfrak{N}_\Theta} \cellset.\]  We call the resulting cocontinuous functor \(\CMcal{O}_2:\psh{\Delta} \to \cellset\).  

We will now give a construction for the object \(Z\).  Recall Rezk's definition of \(Z\)  as the colimit of the diagram \[\Delta[3] \xleftarrow{(\delta^{02},\delta^{13})} \Delta[1] \coprod \Delta[1] \to \Delta[0]\coprod \Delta[0].\]  We let \(Z^{(0)} = \Theta[1]\) and let \(Z^{(1)} = \CMcal{O}_2(Z)\).  Then we see that \(Z^{(1)}\) is the colimit of the diagram \[\CMcal{O}_2(\Delta[3]) \xleftarrow{(\delta^{02},\delta^{13})} \Theta[1] \coprod \Theta[1] \to \Theta[0]\coprod \Theta[0].\]

The object \(Z^{(1)}\) is naturally equipped with a pair of maps \(D_1[\Delta[1]] \to Z^{(1)}\), one classifying the \(2\)-cell \(s_0(z_0) \to gf\), and the other classifying the \(2\)-cell \(s_0(z_1)\to fg\).  Then we define \(Z^{(k)}\) recursively to be the colimit of the diagram\[Z^{(1)} \leftarrow D_1[\Delta[1]]\coprod D_1[\Delta[1]] \to D_1[I^{(k-1)}] \coprod D_1[Z^{(k-1)}].  \]  It is easy to see that \(Z^{(k)}\) embeds naturally into \(Z^{(k+1)}\), and moreover, that this embedding is a pushout of a coproduct of maps of the form \(D_{k-1}[\Delta_1] \to D_{k-1}[Z^{(1)}]\).  

\begin{lemma} The map \(\HMap_\Delta(Z^{(1)},X)_{\Loc_1} \to H_\Delta(Z^{(0)},X)_{\Loc_0}\) is a weak equivalence for any stably Segal space \(X\).  
\end{lemma}
\begin{defn} We say that an object \([t]\) of \(\Theta\) is \dfn{primitive} if it is of the form \(\Delta_1^k[\Delta_j[D_i]]\) for \(i,j,k\geq 0\).  That is, if it is the \(k\)-fold suspension of a \(0\)-cocomposite of a family of \(j\) disks of height \(i+1\), or equivalently, if it is the \(k\)-cocomposite of a family of \(j\) disks of height \(i+k+1\).  In particular, we denote such an object \([t]\) by the symbol \([b(i,j,k)]\).    For any primitive object \([b(i,j,k)]\), we define \(\In[b(i,j,k)]\) to be the subobject of \(\Theta[b(i,j,k)]\) consisting of the \(j+1\) \(k\)-cells, or equivalently, the smallest subobject of \(\Theta[b(i,j,k)]\) containing all of the cells of \([b(i,j,k)]\) of height at most \(k\).  
\end{defn}

Let \([b(i,j,k)]\) be a primitive object of \(\Theta\). We have apparent inclusions \(\In[b(i,j,k)\subseteq \Sp[b(i,j,k)] \subseteq \Theta[b(i,j,k)]\).  Moreover, we have a map \([b(i,1,k)]=D_{i+1+k}\to [b(i,j,k)]\) given by the inclusion of the cospine.  This carries \(\In[b(i,1,k)\) into \(\In[b(i,j,k)]\) in the obvious way. If \(X\) is a stably Segal space, a map \(\In[b(i,j,k)]\to X\) gives a family of \(j+1\) parallel \(k\)-cells, \((f_0,\dots,f_ j)\) in \(X\).  Then we define \[\map^i_X(f_0,\dots,f_j) =\varprojlim(\HMap_\Delta(\Theta[b(i,j,k),X) \to \HMap_\Delta(\In[b(i,j,k)],X) \leftarrow \Delta[0]).\]  Then we notice that there is a trivial fibration \[\map^i_X(f_0,\dots,f_j) \to \varprojlim(\HMap_\Delta(\Sp[b(i,j,k),X) \to \HMap_\Delta(\In[b(i,j,k)],X) \leftarrow \Delta[0]).\]  However, this limit is easily seen to be isomorphic to \[\map^i_X(f_0,f_1)\times \cdots \times \map^i_X(f_{j-1},f_j).\]  Additionally we obtain a map \[\map^i_X(f_0,\dots,f_j)\to \map^i_X(f_0,f_j)\] induced by the inclusion of the cospine as discussed above.  Then this gives a composition map \[\map^i_X(f_0,f_1)\times \cdots \times \map^i_X(f_{j-1},f_j) \to \map^i_X(f_0,f_j)\] in the homotopy category of simplicial sets.  



\begin{lemma} 
 If \(X\) is a stably Segal simplicial cellular set, the explicit embedding \(\Theta[1]\hookrightarrow I\) of one of the two generating \(1\)-cells of \(P\) determines a morphism of simplicial sets \(\delta:\HMap_\Delta(P, X)\to \HMap_\Delta(\Theta[1],X) \cong X[1]\).  Moreover, we have a canonical embedding of \(X[\operatorname{equiv}]\) into \(X[1]\).  Additionally, by the construction of \(I\) (using polygraphs), we see that a map \((I\times \Delta[n] \to X\) at least determines the data for our chosen generating \(1\)-cell to be an equivalence.  This implies that the image of \(\delta\) lies in \(X[\operatorname{equiv}]\).  We claim, then, that this induced map \(\gamma:\HMap_\Delta(I,X)\to X[\operatorname{equiv}]\) is a weak equivalence of simplicial sets.
\end{lemma}
We defer the proof of this lemma until later.



With this lemma in hand, we now notice that our original question reduces to a question about whether or not \((X^{D_n})[0] \to (X^{D_n})[\operatorname{equiv}]\) is a weak homotopy equivalence.  Ultimately, following a modified and generalized version of the proof of \cite{rezk-segal-spaces}{12.1} by Rezk, we obtain that the \(p\)-local stably Segal simplicial cellular sets form a cartesian-closed model category.  Then using our suspension-stabilization procedure described earlier, we may attach all of the suspended copies of the map \(P\to \Theta[0]\).  This gives a completely new model structure not yet described anywhere else in the literature, and it appears to be the case that this model structure finally accepts the embedding from the model category of strict \(\omega\)-categories, and this embedding preserves weak equivalences. 

%There are some problems that need to be worked out regarding this model structure, and ultimately, it doesn't introduce any really interesting or new ideas.  Ultimately, we have always been troubled by the strange procedure of localizing at the suspensions of what we understand to be our ``cylinder'' .  The next major revision of this paper will include a new ``single-step'' procedure for the generation of higher-order invertible structure data.  This new procedure involves the lax tensor product and uses the fact that an \(\omega\)-equivalence between parallel n-cells can be represented by an \(\omega\)-functor out of the tensor product with \(P\).  

%Ultimately, the next revision of this paper should include a comparison between the two models we have, as well as perhaps finding certain new and interesting applications for the lax tensor product and the lax join.  

\to \Delta\Hom(\Theta[1],X) = X[1]\) induced by the inclusion \(\Theta[1]\hookrightarrow P\) factors through the inclusion  \(X_{\operatorname{equiv}}\subseteq X[1]\).  Moreover, the induced map \(\Delta\Hom(P\times \Delta_0, X)\to X_{\operatorname{equiv}}\) is a weak homotopy equivalence.

We given a family \(x_0,\dots, x_n\) of vertices in \(X[0]\), we define the simplicial hom-object \(X_\Delta(x_0,\dots,x_n)\) to be the limit  \[\varprojlim(X[n]\xrightarrow{\nu_0,\dots,\nu_n} \prod_{i=0}^n X[0] \xleftarrow{x_0,\dots,x_n} \Delta[0]).\]  We know that the canonical map \(X[n] \to \Delta\Hom(\Sp[n], X)\) must be a trivial fibration, since \(X\) is stably Segal. The fibre of the induced map \(\Delta\Hom(\Sp[n],X)\to \prod_{i=0}^n X[0] \leftarrow \Delta[0] \) is given precisely by \(X_\Delta(x_0,x_1)\times \dots \times X_\Delta(x_{n-1}, x_n\). This gives a trivial fibration \(X_\Delta(x_0,\dots,x_n)\to X_\Delta(x_0, x_1) \times \dots \times X_\Delta(x_{n-1} x_n).\)  

\begin{thm} If \(X\) is a stably Segal simplicial cellular set, then the map on homotopy function complexes \(\Delta\Hom(P \times \Delta_0, X) \to \Delta\Hom(\Theta[1],X) = X[1]\) induced by the inclusion \(\Theta[1]\hookrightarrow P\) factors through the inclusion  \(X_{\operatorname{equiv}}\subseteq X[1]\).  Moreover, the induced map \(\Delta\Hom(P\times \Delta_0, X)\to X_{\operatorname{equiv}}\) is a weak homotopy equivalence.
\end{thm}    
\begin{proof}
\end{proof}